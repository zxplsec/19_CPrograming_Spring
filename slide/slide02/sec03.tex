\section{使程序可读的技巧}
\begin{frame}[fragile]\ft{提高程序可读性}
\begin{itemize}
\item 变量命名时做到“见其名知其意”;\\[0.1in]
\item 合理使用注释;\\[0.1in]
\item 使用空行分隔一个函数的各个部分,如声明、操作等;\\[0.1in]
\item 每条语句用一行。注意,C允许把多条语句放在同一行或一条语句放多行。
\end{itemize}
\end{frame}


\begin{frame}[fragile]\ft{提高程序可读性}
\begin{lstlisting}[
language=c,
frame=single,
numbers=left
]
// mile_km.c: Convert 2 miles to kilometers
#include<stdio.h>
int main(void) 
{
  float mile, km;	
  mile = 2;
  km = 1.6 * mile;
  printf("%d mile = %d km\n", mile, km);
  printf("Yes, %d km\n", 1.6 * mile);
  return 0;
}
\end{lstlisting}
\end{frame}


\begin{frame}[fragile]\ft{程序说明}
建议在程序开始处用一个注释说明文件名和程序的作用。该过程花不了多少时间,但对以后浏览或打印程序很有帮助。
\end{frame}

\begin{frame}[fragile]\ft{多个声明}
\begin{lstlisting}[
language=c,
frame=single,
numbers=left
]
  float mile, km;
\end{lstlisting}
等同于
\begin{lstlisting}[
language=c,
frame=single,
numbers=left
]
  float mile;
  float km;
\end{lstlisting}
\end{frame}

\begin{frame}[fragile]\ft{输出多个值}
第一个printf语句用了两个占位符:第一个\%d为mile占位,第二个\%d为km占位;圆括号中有三个参数,之间用逗号隔开。\vspace{0.1in}

第二个printf语句说明输出的值可以是一个表达式。
\end{frame}
