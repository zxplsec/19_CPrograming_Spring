\section{程序解释}
\begin{frame}\ft{\#include指示和头文件}

  % \lstinputlisting[
  % language=c,
  % linerange={2-2},
  % firstnumber=2,
  % numbers=left,
  % frame=tb
  % ]{Code/first.c}

  \begin{itemize}
  \item
    相当于在此处复制文件 \lstinline|stdio.h| 的完整内容,以方便在多个程序间共享公用信息。\\[0.1in]
  \item
    \lstinline|\#include| 语句是C预处理器指令的一个例子。通常,C编译器在编译前要对源代码做一些准备工作,这称为预处理。
    \\[0.1in]
  \item
    \lstinline|stdio.h|文件包含了有关输入和输出函数的信息,以供编译器使用。
  \end{itemize}
\end{frame}

\begin{frame}\ft{main函数}
  % \lstinputlisting[
  % language=c,
  % linerange={3-3},
  % firstnumber=3,
  % numbers=left,
  % frame=tb
  % ]{Code/first.c}
  
  C程序至少包含一个函数,函数是C程序的基本模块。\\[0.1in]
  \begin{table}
    \centering
    \begin{tabular}{ll}
      表达式 & 含义 \\ \hline
      \lstinline|( )| & \lstinline|main| 为函数名\\[0.1in]
      \lstinline|int| & \lstinline|main()| 返回一整数\\[0.1in]
      \lstinline|void| & \lstinline|main()| 不接受任何参数 \\ \hline
    \end{tabular}
  \end{table}
  
  注:\lstinline|main()| 是任何C程序的唯一入口。
\end{frame}

\begin{frame}[fragile]\ft{main函数}
  \lstinline|main()| 有三种定义:
  \begin{lstlisting}[language=c,frame=single]
void main()
{
  /* Definition 1: NOT RECOMMENDED */
}

int main()
{
  /* Definition 2 */
}

int main(int argc, char* argv[])
{
  /* Definition 3 */
}
  \end{lstlisting}
\end{frame}

\begin{frame}[fragile]\ft{main函数}
  考虑main()的两种定义,它们的差别是什么?
  \begin{minipage}{0.45\linewidth}
  \begin{lstlisting}
int main()
{
  /* ... */
  return 0;
}
  \end{lstlisting}
  \end{minipage}\hfill
  \begin{minipage}{0.45\linewidth}
  \begin{lstlisting}
int main(void)
{
  /* ... */
  return 0;
}
  \end{lstlisting}
    
  \end{minipage}
% \end{frame}

% \begin{frame}[fragile]\ft{main函数}
  \pause
  \begin{itemize}
  \item  在C++中,两种定义没有差别,完全一致。\\[.1in]
  \item  在C中,两种定义都可以,但是第二种定义更好,因它清晰地表明 \lstinline|main()| 在调用时不接受任何参数。    
  \end{itemize}
  \pause 
  \begin{free}[注意]{}
    在C中,如果一个函数在定义时没有指定任何参数,就意味着在调用该函数时允许接受任意多个参数或不接受参数。
  \end{free}
\end{frame}

\begin{frame}[fragile]\ft{main函数}
  \begin{lstlisting}[language=c,frame=single]
// Program 1:
//   Compiles and runs fine in C, but not in C++
void fun() {  } 
int main(void)
{
  fun(10, "GfG", "GQ");
  return 0;
}
  \end{lstlisting}
  \pause
  \begin{lstlisting}[language=c,frame=single]
// Program 2
// Fails in compilation in both C and C++
void fun(void) {  }
int main(void)
{
  fun(10, "GfG", "GQ");
  return 0;
}
  \end{lstlisting}
\end{frame}


\begin{frame}\ft{注释}
  
  % \lstinputlisting[
  % language=c,
  % linerange={4-4},
  % firstnumber=4,
  % numbers=left,
  % frame=tb
  % ]{Code/first.c}
  


  \begin{overprint}
    \begin{itemize}
    \item {\tf /* ... */}之间的内容是程序注释。\\[0.1in]
    \item 注释可以让阅读者更容易理解程序。\\[0.1in]
    \item 注释可以放在任意位置,甚至和它要解释的语句在同一行。\\[0.1in]
    \item 一个较长的注释可以单独放一行,也可以是多行。\\[0.1in]
    \item {\tf /* ... */}之间的所有内容都会被编译器忽略。
    \end{itemize}
  \end{overprint}

\end{frame}

\begin{frame}[fragile]\ft{注释}
  \begin{lstlisting}[language=c]
/* valid comment  */

/* vomment can be seperated 
   into multiple lines */

/*
   valid comment
*/

/* invalid comment
  \end{lstlisting}

  \begin{lstlisting}[language=c]
// Such a comment must restricted in one line

int n; // comment can be here
  \end{lstlisting}
\end{frame}

\begin{frame}[fragile]\ft{花括号,程序体和代码块}
  
  \begin{lstlisting}[language=c,frame=tb]
{
  ...
}
  \end{lstlisting}
  
  \begin{itemize}
  \item
    C函数使用花括号表示函数体的开始和结束。\\[0.2in]
  \item
    花括号还可以用来把函数中的语句聚集到一个单元或代码块中。
  \end{itemize}
\end{frame}

\begin{frame}[fragile]\ft{声明}
  % \lstinputlisting[
  % language=c,
  % linerange={6-6},
  % firstnumber=6,
  % numbers=left,
  % frame=tb
  % ]{Code/first.c}

  
  该语句为声明语句(declaration statement),做两件事情:\vspace{0.1in}
  \begin{itemize}
  \item[(1)]
    在内存中为变量 \lstinline|num| 分配了空间。\\[0.1in]
  \item[(2)]
    \lstinline|int| 说明变量 \lstinline|num| 的类型(整型)。
  \end{itemize} \vspace{0.1in}

  \pause 

  注意:分号指明该行是C的一个语句。分号是语句的一部分。
  
\end{frame}

\begin{frame}[fragile]\ft{声明}
  Ansi C要求必须在一个代码块的开始处声明变量,在这之前不允许其他任何语句。

  \begin{lstlisting}[
    language=c,
    numbers=left,
    frame=tb
    ]
    int main(void)
    {
      int n;
      int m;
      n = 5;
      m = 3;
      // other statements
    }
  \end{lstlisting}
\end{frame}

\begin{frame}[fragile]\ft{声明}
  C99遵循C++的惯例,允许把声明放在代码块的任何位置。但是在首次使用变量之前仍必须先声明它。

  \begin{lstlisting}[
language=c,
numbers=left,
frame=tb
]
int main(void)
{
  int n;
  n = 5;
  // more statements
  int m;
  m = 3;
  // other statements
}
  \end{lstlisting}
\end{frame}


\begin{frame}[fragile]\ft{声明}
  \begin{question}[]{}
    \begin{itemize}
    \item 数据类型是什么?
    \item 可以选择什么样的名字?
    \item 为什么必须对变量进行声明?
    \end{itemize}
  \end{question}
\end{frame}

\begin{frame}[fragile]\ft{声明}
  \begin{enumerate}[1]
  \item 数据类型\\[0.1in]
    \begin{itemize}
    \item C可以处理多种数据类型,如整数、字符和浮点数。\\[0.1in]
    \item \red{把一个变量声明为整数类型、字符类型或浮点数类型,是计算机正确地存储、获取和解释该数据的基本前提。}
    \end{itemize}
  \end{enumerate}
% \end{frame}

% \begin{frame}[fragile]\ft{声明}
  \begin{enumerate}[2]
  \item 如何命名?\\[0.1in]
    \begin{itemize}
    \item 应尽量使用有意义的变量名。\\[0.1in]
    \item 若名字不能表达清楚,可以用注释解释变量所代表的意思。\\[0.1in]
    \item 通过这些方式使程序更易读是良好编程的基本技巧之一。 
    \end{itemize}
  \end{enumerate}
\end{frame}


\begin{frame}[fragile]\ft{声明}
  \begin{free}[命名规则]{} 
      \begin{enumerate}
      \item 只能使用字母、数字和下划线,且第一个字符不能为数字。 \\[0.1in]
      \item 操作系统和C库通常使用以一个或两个下划线开始的名字,因此最好避免这种用法。 \\[0.1in]
      \item C区分大小写,如 \lstinline|stars| 不同于 \lstinline|Stars| 或 \lstinline|STARS|.
      \end{enumerate}
    \end{free}
% \end{frame}


% \begin{frame}[fragile]\ft{声明}

  \begin{table}
    \centering
    % \caption{正确和错误的名字}
    \begin{tabular}{c|c}\hline
      Yes& No\\[0.1in]\hline
      \lstinline|wiggles|  &  \lstinline|$zj**|\\[0.1in]
      \lstinline|cat2|     &  \lstinline|2cat|\\[0.1in]
      \lstinline|Hot_Dog|  &  \lstinline|Hot-Dog|\\[0.1in]
      \lstinline|taxRate|  &  \lstinline|tax Rate|\\[0.1in]
      \lstinline|_kcab|    &  \lstinline|don't|\\\hline
    \end{tabular}
  \end{table}
\end{frame}


\begin{frame}[fragile]\ft{声明}
  \begin{enumerate}[3]
  \item 声明变量的好处\\[0.1in]
    \begin{itemize}
    \item 把所有变量放在一起,可以让读者更容易掌握程序的内容。
      \\[0.1in]
    \item 在开始编写程序之前,考虑一下需要声明的变量会促使你做一些计划。\\[0.1in]
    \item 声明变量可以帮助避免程序中出现一类很难发现的细微错误,即变量名的错误拼写。\\[0.1in]
    \item 若没有声明所有变量,将不能编译C程序。

    \end{itemize}
  \end{enumerate}
\end{frame}


\begin{frame}[fragile]\ft{赋值}
  % \lstinputlisting[
  % language=c,
  % linerange={7-7},
  % firstnumber=7,
  % numbers=left,
  % frame=tb
  % ]{Code/first.c}

  
  该语句是赋值语句(Assignment statement)。赋值语句是C语言的一种基本操作。\vspace{0.1in}

  含义:把值赋给变量 \lstinline|num|。
  
\end{frame}


\begin{frame}[fragile]\ft{printf函数}
  % \lstinputlisting[
  % language=c,
  % linerange={8-10},
  % firstnumber=8,
  % numbers=left,
  % frame=tb
  % ]{Code/first.c}
  

  每行都使用了C的一个标准函数 \lstinline|printf()|,其信息由头文件 \lstinline|stdio.h| 指定。
  \vspace{0.1in}

  圆括号 \lstinline|()| 表明 \lstinline|printf| 为函数名,圆括号内为参数(argument)。这里的参数都是字符串,即双引号之间的内容。
  
\end{frame}


\begin{frame}[fragile]\ft{转义字符}
      
  转义字符通常用于代表难以表达或无法键入的字符,以 \lstinline|\| 开头。
  
  \begin{table}
    \centering
    \begin{tabular}{cc} \hline
      转移字符 & 含义 \\ \hline  
      \lstinline|\n| & 换行\\
      \lstinline|\t| & Tab键\\
      \lstinline|\b| & 退格\\
      \lstinline|\'| & 单引号\\
      \lstinline|\"| & 双引号\\
      \lstinline|\\| & 反斜杠\\
      \hline 
    \end{tabular}
  \end{table}
\end{frame}


\begin{frame}[fragile]\ft{格式化字符串}
  格式化字符串,也称占位符,用以指定输出项的数据类型和输出格式,
  以 \lstinline|%| 开头。

  \begin{table}
    \centering
    \begin{tabular}{p{2cm}|p{8cm}}\hline
      占位符 & 含义 \\ \hline  
      \lstinline|%d| & 用于输出十进制整数(实际长度)\\[0.1in]
      \lstinline|%c| & 输出一个字符\\[0.1in]
      \lstinline|%s| & 输出一个字符串\\[0.1in]
      \lstinline|%f| & 以小数形式输出实数(整数部分全部输出,小数部分6位)\\
      \hline 
    \end{tabular}
  \end{table}
  
\end{frame}

\begin{frame}[fragile]\ft{return语句}
  % \lstinputlisting[
  % language=c,
  % linerange={11-11},
  % firstnumber=11,
  % numbers=left,
  % frame=tb
  % ]{Code/first.c}

  带有返回值的C函数要求使用一个 \lstinline|return| 语句,
  该语句包含关键字 \lstinline|return|,后面紧跟要返回的值。

\end{frame}
