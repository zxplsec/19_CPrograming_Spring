\section{使用C语言的七个步骤}


\begin{frame}\ft{\secname}
  
  \begin{itemize}
  \item[(1)] 定义程序目标\\[0.1in]
  \item[] 开始时应对程序做什么有一个清晰的想法。\\[0.1in]
  \item[] 考虑程序需要的信息、需要进行的计算和操作,以及程序应该向你报告的信息。
  \end{itemize}
  
\end{frame}


\begin{frame}\ft{\secname}
  
  \begin{itemize}
  \item[(2)] 设计程序\\[0.1in]
  \item[] 如何表示数据   \\[0.1in]
  \item[] 用什么方法处理数据  \\[0.1in]
  \item[] \red{选择一个好的方式来表示信息可以使程序设计和数据处理容易很多}
  \end{itemize}

\end{frame}


\begin{frame}\ft{\secname}
  
  \begin{itemize}
  \item[(3)] 编写代码\\[0.1in]
  \item[] 用文本编辑器创建一种称为源代码的文件。
  \item[] 常见的文本编辑器Ultraedit、Emacs、Vi、集成开发环境(IDE)中自带的编辑器等。
  \end{itemize}
  
\end{frame}


\begin{frame}\ft{\secname}
  \begin{itemize}
  \item[(4)] 编译 \\[0.1in]
  \item[] 编译器将源代码转换为可执行文件,可执行文件是机器语言表示代码。\\[0.2in]
  \item[(5)] 运行程序 \\[0.1in]
  \item[] 在Dos、Unix和Linux系统中,可在命令行中直接键入可执行文件名即可运行程序。\\[0.1in]
  \item[] 在Windows和MAC环境提供的IDE中可通过选择菜单选项或特定快捷键来执行程序。 
  \end{itemize} 
  
\end{frame}

\begin{frame}[fragile]\ft{\secname}
  \begin{itemize}
  \item[(6)] 测试和调试程序\\[0.1in]
  \item[] Bug\\[0.1in]
  \item[] Debug\\[0.2in]

  \item[(7)]  维护和修改程序\\[0.1in]
  \end{itemize} 
  
\end{frame}



