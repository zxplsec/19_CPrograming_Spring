\section{计算机的基本软件组成}

\begin{frame}\ft{\secname}
软件是组成计算机系统的重要部分,分为系统软件和应用软件两大类。 
\end{frame}

\begin{frame}\ft{\secname}
\begin{itemize}
\item 系统软件是由计算机生产厂商为使用该计算机而提供的基本软件。如操作系统、文字处理程序、计算机语言处理程序、数据库管理程序等。\\[0.1in]
\item 应用软件是指用户为了自己的业务应用而使用系统开发出来的用户软件。如音频视频播放器、QQ、微信等。
\end{itemize}
系统软件依赖于机器,而应用软件则更接近用户业务。
\end{frame}

\begin{frame}\ft{操作系统(OS)}
\begin{itemize}
\item 操作系统是最基本也是最重要的系统软件。\\[0.1in]
\item 它负责管理计算机系统的各种硬件资源,如CPU、内存空间、磁盘空间、外部设备等,并且负责解释用户对机器的管理命令,使它转换为机器实际的操作。\\[0.1in]
\item 常见的操作系统:DOS、 WINDOWS、 UNIX(LINUX)、OS X等。
\end{itemize}
\end{frame}

\begin{frame}\ft{计算机语言处理程序}
计算机语言分为机器语言、汇编语言和高级语言。 

\begin{itemize}
\item 机器语言:机器能直接认识的语言,是由“1”和“0”组成的一组代码指令。\\[0.1in]
\item 汇编语言:实际是由一组与机器语言指令一一对应的符号指令和简单语法组成的。\\[0.1in]
\item 高级语言:比较接近日常用语,对机器依赖性低,即适用于各种机器的计算机语言。如Basic、Visual Basic、Fortran、C/C++、Java、Python等。
\end{itemize}
\end{frame}

\begin{frame}\ft{计算机语言处理程序}

将高级语言翻译为机器语言,有两种方式,一种叫“编译”,一种叫“解释”。 \vspace{0.05in}

\begin{itemize}
\item
  \red{编译型语言}将程序作为一个整体进行处理,编译后与子程序库连接,形成一个完整的可执行程序。
  \begin{itemize}
  \item  优点:可执行程序运行速度很快
  \item  缺点:编译、链接比较费时
  \item 常见语言: Fortran、C/C++语言
  \end{itemize}
\item
  \red{解释型语言}则对高级语言程序逐句解释执行。
  \begin{itemize}
  \item 优点:程序设计的灵活性大
  \item 缺点:运行效率较低
  \item 常见语言:Basic、Python、Matlab
  \end{itemize}
\end{itemize}
\end{frame}
