\section{格式化输入}
\begin{frame}[fragile]\ft{\secname}
\tf 同printf()一样,scanf()也使用控制字符串和参数列表,主要区别在参数列表。printf()使用变量名、常量和表达式;而scanf()使用指向变量的指针。
\end{frame}

\begin{frame}[fragile]\ft{\secname:scanf()的参数列表}
\begin{itemize}
\item 
\tf 若使用scanf()来读取某种基本类型的值,请在变量名前加一个\&。 \\[0.1in]
\item
若使用scanf()把一个字符串读入一个字符数组,请不要使用\&。
\end{itemize}
\end{frame}

\begin{frame}[fragile]\ft{\secname:scanf()的参数列表}
\lstinputlisting[language=c,frame=single,numbers=left]{Code/input.c}
\end{frame}

\begin{frame}[fragile]\ft{\secname:scanf()的参数列表}
\begin{lstlisting}[showspaces=true,backgroundcolor=\color{red!20}]
$ gcc input.c
$ ./a.out
Enter your name, age and weight:
Xiaoming
23 100
Xiaoming: 23 100.000000
\end{lstlisting}
\end{frame}

\begin{frame}[fragile]\ft{\secname:scanf()的参数列表}
\begin{itemize}
\item
\tf scanf()使用空格(换行、制表符和空格)来决定如何把输入分成几个字段。
它依次把格式说明符与字段相匹配,并跳过它们之间的空格。\\[0.1in]
\item
也可以分一行或多行输入,只要每个输入项目之间至少有一个换行符、空格或制表符。\\[0.1in]
\item
\%c是个例外,即使下一个字符是空白字符,它也会读取。
\end{itemize}
\end{frame}

\begin{frame}[fragile]\ft{\secname:printf与scanf格式说明符的区别}
\begin{itemize}
\item
\tf printf()把\%f、\%e、\%E、\%g和\%G同时用于float和double类型\\[0.1in]
\item
scanf()只把它们用于float类型,而用于double类型时要求加上l修饰符。
\end{itemize}
\end{frame}

\begin{frame}[fragile]\ft{\secname:scanf格式说明符}
\begin{table}
\centering
\begin{tabular}{p{3cm}|p{7cm}}\hline
格式说明符 & 意义 \\\hline\hline
\tf\%c & 把输入解释成一个字符 \\[2mm]
\tf\%d & 把输入解释成一个有符号十进制数 \\[2mm]
\tf\%e,\%f,\%g,\%a & 把输入解释成一个浮点数\\[2mm]
\tf\%E,\%F,\%G,\%A & 把输入解释成一个浮点数\\[2mm]
\tf\%i & 把输入解释成一个有符号十进制数\\[2mm]
\tf\%o & 把输入解释成一个有符号八进制数\\[2mm]
\tf\%p & 把输入解释成一个指针\\ \hline
\end{tabular}
\end{table}
\end{frame}

\begin{frame}[fragile]\ft{\secname:scanf格式说明符}
\begin{table}
\centering
\begin{tabular}{p{2cm}|p{8cm}}\hline
格式说明符 & 意义 \\\hline\hline
\tf\%s & 把输入解释成一个字符串:输入内容以第一个非空白字符作为开始,并且包含到下一个空白字符的全部字符 \\[2mm]
\tf\%u & 把输入解释成一个无符号十进制数 \\[2mm]
\tf\%x,\%X & 把输入解释成一个有符号十六进制数\\ \hline
\end{tabular}
\end{table}
\end{frame}

\begin{frame}[fragile]\ft{\secname:scanf格式说明符}
\tf 可在格式说明符中使用修饰符,修饰符出现在\%与格式字符之间。
\end{frame}

\begin{frame}[fragile]\ft{\secname:scanf格式说明符}

\begin{table}
\centering
\begin{tabular}{p{2cm}|p{8cm}}\hline
修饰符 & 意义 \\\hline\hline
\tf * &  滞后赋值,如{"\tf \%*d"} \\[2mm]
\tf digit & 最大字段宽度:在达到最大字段宽度或遇到第一个空白字符时停止对输入项的读取,如{\tf "\%10s"} \\[2mm]
\tf hh & 把整数读作{\tf signed char}或{\tf unsigned char},如{\tf "\%hhd"}或{\tf "\%hhu"} \\[2mm]
\tf ll & 把整数读作{\tf long long}或{\tf unsigned long long},如{\tf "\%lld"}或{\tf "\%llu"}\\
\hline
\end{tabular}
\end{table}
\end{frame}

\begin{frame}[fragile]\ft{\secname:scanf格式说明符}
\begin{table}
\centering
%\caption{修饰符h, l 或 L}
\begin{tabular}{p{3.5cm}|p{6cm}}\hline
修饰符 & 意义 \\\hline\hline
\tf "\%hd", "\%hi" & 以{\tf short}存储 \\[2mm]\hline
\tf "\%ho", "\%hx", "\%hu" & 以{\tf unsigned short}存储\\[2mm]\hline
\tf "\%ld", "\%li" & 以{\tf long}存储\\[2mm]\hline
\tf "\%lo", "\%lx", "\%lu" & 以{\tf unsigned long}存储\\[2mm]\hline
\tf "\%le", "\%lf", "\%lg" & 以{\tf double}存储
\\[2mm]\hline
\tf "\%Le", "\%Lf", "\%Lg" & 以{\tf long double}存储 \\
\hline
\end{tabular}
\end{table}
\end{frame}

\begin{frame}[fragile]\ft{\secname:scanf格式说明符}
\tf 若没有这些修饰符,则\%d, \%i, \%o和\%x指示int类型,而\%e, \%f和\%g指示float类型。
\end{frame}

\begin{frame}[fragile]\ft{\secname:格式字符串中的常规字符}
\tf scanf()允许把普通字符放在格式字符串中,除了空格字符之外的普通字符一定要与输入字符串准确匹配。  
\end{frame}

\begin{frame}[fragile]\ft{\secname:格式字符串中的常规字符}
\begin{lstlisting}[showspaces=true,backgroundcolor=\color{red!20}]
scanf("%d, %d", &n, &m);
\end{lstlisting}

\rule{\textwidth}{0.1em}
\begin{lstlisting}[title=合法的输入方式,showspaces=true,backgroundcolor=\color{red!20}]
12, 23

12,     23

12 , 23

12,
23
\end{lstlisting}

\end{frame}

\begin{frame}[fragile]\ft{\secname:格式字符串中的常规字符}
\begin{lstlisting}[showspaces=true,backgroundcolor=\color{red!20}]
scanf("%d and %d", &n, &m);
\end{lstlisting}

\rule{\textwidth}{0.1em}
\begin{lstlisting}[title=合法的输入方式,showspaces=true,backgroundcolor=\color{red!20}]
12 and 23

12 and     23

12and23
\end{lstlisting}
\end{frame}

\begin{frame}[fragile]\ft{\secname:格式字符串中的常规字符}

除了{\tf \%c}之外的说明符会自动跳过输入项之前的空格,故以下两条语句的效果相同:
\begin{lstlisting}[showspaces=true,backgroundcolor=\color{red!10}]
scanf("%d%d",&n,&m);

scanf("%d %d",&n,&m);
\end{lstlisting}
\end{frame}

\begin{frame}[fragile]\ft{\secname:格式字符串中的常规字符}
对于{\tf \%c}来说,向格式字符串中添加一些空格将导致一些差别。如:
\begin{lstlisting}[showspaces=true,backgroundcolor=\color{red!10}]
scanf("%c",&ch)
\end{lstlisting}
读取在输入中遇到的第一个字符,而
\begin{lstlisting}[showspaces=true,backgroundcolor=\color{red!10}]
scanf(" %c",&ch)
\end{lstlisting}
则读取遇到的第一个非空白字符。
\end{frame}

\begin{frame}[fragile]\ft{\secname:scanf的返回值}
{\tf scanf()}返回成功读入的项目个数。

\begin{itemize}
\item
\tf 若没有读取任何项目,则返回0;
\item
若检测到文件结尾(end of file),则返回EOF。
(EOF是stdio.h中定义的特殊值,一般为-1)
\end{itemize}

\end{frame}

\begin{frame}[fragile,allowframebreaks]\ft{\secname:printf()的*修饰符}
\lstinputlisting[language=c,frame=single,numbers=left]{Code/varwidth.c}
\end{frame}

\begin{frame}[fragile]\ft{\secname:printf的*修饰符}
\begin{lstlisting}[showspaces=true,backgroundcolor=\color{red!10}]
$ gcc varwidth.c
$ ./a.out
What field width?
6
The number is:   256
Now enter a width and a precision:
8 3
Weight= 123.500
\end{lstlisting}
\end{frame}

\begin{frame}[fragile]\ft{\secname:scanf()的*修饰符}
\tf 在scanf()中,把*放在\%与格式字符之间时,会使函数跳过相应的输入项目。
\end{frame}

\begin{frame}[fragile]\ft{\secname:scanf()的*修饰符}

\lstinputlisting[language=c,frame=single,numbers=left]{Code/skip2.c}
\end{frame}

\begin{frame}[fragile]\ft{\secname:scanf()的*修饰符}
\begin{lstlisting}[showspaces=true,backgroundcolor=\color{red!10}]
$ gcc skip2.c
$ ./a.out
Please enter three integers:
10 20 30
The last integer was 30
\end{lstlisting}
\end{frame}





