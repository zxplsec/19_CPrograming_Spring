\section{递归}

\begin{frame}[fragile]\ft{\secname}
C允许一个函数调用其自身,这种调用过程被称为递归(recursion)。 
\vspace{0.1in}

\begin{itemize}
\item 递归一般可用循环代替。有些情况使用循环会比较好,而有时使用递归更有效。\\[0.1in]
\item 递归虽然可使程序结构优美,但其执行效率却没循环语句高。
\end{itemize}
\end{frame}


\begin{frame}[fragile,allowframebreaks]\ft{\secname}
  \lstinputlisting
  [language=c,numbers=left,frame=single]
  {Code/recur.c}
\end{frame}


\begin{frame}[fragile]\ft{\secname}
\begin{lstlisting}[backgroundcolor=\color{red!10}]
Level 1: n location 0x7fff5fbff7bc
Level 2: n location 0x7fff5fbff79c
Level 3: n location 0x7fff5fbff77c
Level 4: n location 0x7fff5fbff75c
LEVEL 4: n location 0x7fff5fbff75c
LEVEL 3: n location 0x7fff5fbff77c
LEVEL 2: n location 0x7fff5fbff79c
LEVEL 1: n location 0x7fff5fbff7bc
\end{lstlisting}
\end{frame}


\begin{frame}[fragile]\ft{\secname}
{\tf \&}为地址运算符,{\tf \&n}表示存储n的内存地址,{\tf printf()}使用占位符{\tf \%p}来指示地址。
\end{frame}


\begin{frame}[fragile]\ft{\secname:程序分析}
\begin{itemize}
\item 首先,{\tf main()}使用实参1调用{\tf up\_and\_down()},打印语句{\tf \#1}输出{\tf Level 1}。\\[0.1in]
\item 然后,由于$n<4$,故{\tf up\_and\_down()}(第1级)使用实参2调用{\tf up\_and\_down()}(第2级),打印语句{\tf \#1}输出{\tf Level 2}。\\[0.1in]
\item 类似地,下面的两次调用打印{\tf Level 3}和{\tf Level 4}。
\end{itemize}
\end{frame}

\begin{frame}[fragile]\ft{\secname:程序分析}
\begin{itemize}
\item 当开始执行第4级调用时,$n$的值为4,故if语句不满足条件,不再继续调用{\tf up\_and\_down()},接着执行打印语句{\tf \#2},输出{\tf Level 4},至此第4级调用结束,把控制返回给第3级调用函数。\\[0.1in]
\item 第3级调用函数中前一个执行过的语句是在if语句中执行第4级调用,因此,它开始执行后续代码,即执行打印语句{\tf \#2},输出{\tf Level 3}。\\[0.1in]
\item 当第3级调用结束后,第2级调用函数开始继续执行,输出{\tf Level 2}。以此类推。
\end{itemize}
\end{frame}

\begin{frame}[fragile]\ft{\secname:递归的基本原理}
\begin{itemize}
\item
\textcolor{acolor1}{每一级的递归都使用其私有变量n。}\\[0.1in]
\item
每一次函数调用都会有一次返回。当程序执行到某一级递归的结尾处时,它会转移到前一级递归继续执行。
\end{itemize}
\end{frame}

\begin{frame}[fragile]\ft{\secname:递归的基本原理}
\begin{itemize}
\item
递归函数中,位于递归调用前的语句和各级被调函数具有相同的执行次序。\\[0.1in]
\item[] 如打印语句\#1位于递归调用语句之前,它按递归调用的顺序执行4次,即依次为第1级、第2级、第3级和第4级。\\[0.1in]
\item 
递归函数中,位于递归调用后的语句和各级被调函数具有相反的执行次序。\\[0.1in]
\item[] 
如打印语句{\tf \#2}位于递归调用语句之后,执行次序为:第4级、第3级、第2级和第1级。
\end{itemize}
\end{frame}

\begin{frame}[fragile]\ft{\secname:递归的基本原理}
\begin{itemize}
\item 递归函数中,必须包含可以终止递归调用的语句。
\end{itemize}

\end{frame}

\begin{frame}[fragile]\ft{\secname:尾递归}
最简单的递归方式是\textcolor{acolor1}{把递归调用语句放在函数结尾,return语句之前。}这种形式被称为\textcolor{acolor1}{尾递归(tail recursion)}。尾递归的作用相当于一条循环语句,它是最简单的递归形式。
\end{frame}

\begin{frame}[fragile]\ft{\secname:尾递归}
分别使用循环和尾递归编写函数计算阶乘,然后用一个驱动程序测试它们。
\end{frame}

\begin{frame}[fragile,allowframebreaks]\ft{\secname:尾递归}
  \lstinputlisting
  [language=c,numbers=left,frame=single]
  {Code/factor.c}
\end{frame}


\begin{frame}[fragile]\ft{\secname:尾递归}
\begin{lstlisting}
This program calculates factorials.
Enter a value in the range 0-12 (q to quit):
5
loop:      5! = 120
recursion: 5! = 120
Enter a value in the range 0-12 (q to quit):
10
loop:      10! = 3628800
recursion: 10! = 3628800
Enter a value in the range 0-12 (q to quit):
12
loop:      12! = 479001600
recursion: 12! = 479001600
Enter a value in the range 0-12 (q to quit):
q
Bye.
\end{lstlisting}
\end{frame}

\begin{frame}[fragile]\ft{\secname:尾递归}
{\Large 选用循环还是递归?}\pause 一般来说,选择循环更好一些。
\pause
\vspace{0.1in}

\begin{itemize}
\item 每次递归调用都有自己的变量集合,需要占用较多的内存。每次递归调用需要把新的变量集合存储在堆栈中。\\[0.1in]
\item 每次函数调用都要花费一定的时间,故递归的执行速度较慢。
\end{itemize}
\end{frame}

\begin{frame}[fragile]\ft{\secname:尾递归}
{\Large 那为什么要学习递归呢?}\pause 
\vspace{0.1in}

\begin{itemize}
\item 尾递归非常简单,易于理解。\\[0.1in]
\item 某些情况下,不能使用简单的循环语句代替递归,所以有必要学习递归。
\end{itemize}

\end{frame}

\begin{frame}[fragile]\ft{\secname:递归与反向计算}
编写程序,将一个整数转换为二进制形式。
\end{frame}

\begin{frame}[fragile]\ft{\secname:递归与反向计算}
对于奇数,其二进制形式的末位为1;而对于偶数,其二进制形式的末位为0。于是,\textcolor{acolor1}{对于n,其二进制数的末位为n\%2。}
\begin{lstlisting}[backgroundcolor=\color{red!10}]
628
628%10=8  628/10=62  62%10=2   62/10=6   6%10=6
     8                     2                  6
\end{lstlisting}

\begin{lstlisting}[backgroundcolor=\color{red!10}]
5  
 5%2=1  5/2=2  2%2=0  2/2=1  1%2=1
     1             0             1 
10
10%2=0  10/2=5  5%2=1  5/2=2  2%2=0  
     0              1             0
2/2=1  1%2=1
           1
\end{lstlisting}

\end{frame}

\begin{frame}[fragile]\ft{\secname:递归与反向计算}
规律:
\vspace{0.1in}

\begin{itemize}
\item 
在递归调用之前,计算{\tf n\%2}的值,在递归调用之后输出。\\[0.1in]
\item
为算下一个数字,需把原数值除以2。若此时得出的为偶数,则下一个二进制位为0;若得出的是奇数,则下一个二进制位为1。
\end{itemize}
\end{frame}

\begin{frame}[fragile,allowframebreaks]\ft{\secname:递归与反向计算}
\lstinputlisting
  [language=c,numbers=left,frame=single]
  {Code/binary.c}
\end{frame}



\begin{frame}[fragile]\ft{\secname:递归与反向计算}
\begin{lstlisting}[backgroundcolor=\color{red!10}]
Enter an integer (q to quit):
9
Binary equivalent: 1001
Enter an integer (q to quit):
255
Binary equivalent: 11111111
Enter an integer (q to quit):
1024
Binary equivalent: 10000000000
Enter an integer (q to quit):
q
Done.
\end{lstlisting}
\end{frame}


\begin{frame}[fragile]\ft{\secname:递归的优缺点}
 

\begin{itemize}
\item 优点:
\item[] 
为某些编程问题提供了最简单的解决办法。\\[0.1in]
\item 缺点:
\item[]
一些递归算法会很快地耗尽计算机的内存资源,同时递归程序难于阅读和维护。
\end{itemize}

\end{frame}


\begin{frame}[fragile]\ft{\secname:递归的优缺点}
编写程序,计算斐波那契数列。
$$
\begin{aligned}
&F_1=F_2=1,\\[0.1in]
&F_n=F_{n-1}+F_{n-2}, \quad n=3,4,\cd.
\end{aligned}
$$
\end{frame}


\begin{frame}[fragile]\ft{\secname:递归的优缺点}
\begin{lstlisting}
long Fibonacci(int n)
{
  if (n > 2)
    return Fibonacci(n-1) + Fibonacci(n-2);
  else
    return 1;
}
\end{lstlisting}

该函数使用了双重递归(double recursion),即函数对本身进行了两次调用。这会导致一个弱点。{\Huge What?}
\end{frame}


\begin{frame}[fragile]\ft{\secname:递归的优缺点}
每级调用的变量数会呈指数级增长:
\begin{table}
\centering
\caption{每级调用中变量n的个数}
\begin{tabular}{cc}\hline
Level & number of n\\\hline
$1$ & $1$\\
$2$ & $2$\\
$3$ & $2^2$\\
$4$ & $2^3$\\
$\vd$ & $\vd$ \\
$l$ & $2^{l-1}$\\\hline
\end{tabular}
\end{table}
\end{frame}
