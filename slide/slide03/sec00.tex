\section{例子}

\begin{frame}[fragile]\ft{\secname}
  \begin{question}[]{}
  编制程序,实现华氏温度到摄氏温度的转换。转化公式为
  $$
  C = \frac59 (F-32).
  $$
  其中$F$表示华氏温度,$C$表示摄氏温度。    
  \end{question}
\end{frame}

\begin{frame}[fragile]\ft{\secname}

  \lstinputlisting[language=C,numbers=left,frame=single]{slide03/code/fah2cel.c}
\end{frame}

\begin{frame}[fragile]\ft{\secname}

\begin{lstlisting}[backgroundcolor=\color{red!10}]
$ gcc fah2cel.c 
$ ./a.out
Please input the Fahrenheit temperature:
78
78.00 F = 25.56 C
\end{lstlisting}
\end{frame}


\begin{frame}[fragile]\ft{\secname}
\begin{itemize}
\item 变量声明:
\begin{lstlisting}
float fah, cel;
\end{lstlisting}
float类型可以处理小数。\\[0.1in]
\item 格式说明符 \lstinline|%f| 用于输出浮点型数据。 \lstinline|%.2f| 可以精确控制输出格式,使浮点数显示到小数点后两位。\\[0.1in]
\item 使用 \lstinline|scanf()| 函数为程序提供键盘输入。 \\[.1in]
  \begin{itemize}
  \item \lstinline|%f|指示 \lstinline|scanf()| 从键盘读取一个浮点数;\\[.1in]
  \item \lstinline|&fah| 表示变量 \lstinline|fah| 的地址,指定将输入值赋给变量 \lstinline|fah| 。
  \end{itemize}

\end{itemize}
\end{frame}


\begin{frame}[fragile]\ft{\secname}
  \begin{itemize}
  \item
    该程序的最大特点是\red{交互性},交互性使得程序更加灵活。\\[0.1in]
    \begin{itemize}
    \item 例如,该程序可以输入任意的华氏温度,而不必每次重写。\\[0.2in]
    \end{itemize}
  \item  \lstinline|scanf()| 和  \lstinline|printf()| 使得这种交互成为可能。\\[0.1in]
    \begin{itemize}
    \item  \lstinline|scanf()| 从键盘读取数据并将其传递给程序;\\[0.1in]
    \item  \lstinline|printf()| 则从程序读取数据并将其打印到屏幕。\\[0.1in]
    \item 两者一起使用,就建立起了人机之间的双向通信。
    \end{itemize}
  \end{itemize}

\end{frame}




