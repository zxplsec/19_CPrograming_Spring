\documentclass[10pt]{beamer}
\usepackage[framemethod=TikZ]{mdframed}
\usepackage{pgfpages}
%\setbeameroption{show notes on second screen}
\setbeameroption{show notes}

\usetheme[sectionpage=simple,subsectionpage=simple]{metropolis}
\usepackage{appendixnumberbeamer}

\usepackage{booktabs}
\usepackage[scale=2]{ccicons}

\usepackage{pgfplots}
\usepgfplotslibrary{dateplot}

\usepackage{xspace}
\newcommand{\themename}{\textbf{\textsc{metropolis}}\xspace}

\usepackage{amsmath,amsthm,amssymb,mathdots}
\usepackage{fourier}
\usepackage{multicol}
\usepackage{multirow}
%\usepackage{fontspec}
\usepackage{subfigure}
\usepackage[most]{tcolorbox}
\newcounter{testexample}
\usepackage{xparse}
\usepackage{lipsum}
\usepackage[UTF8,noindent]{ctex}
\usepackage{extarrows}
%\usepackage{courier}
\usepackage{animate}
\usepackage{dcolumn}
\usepackage[]{hyperref}
\usepackage{pgf}
\usepackage{tikz}
\usetikzlibrary{calc}
\usetikzlibrary{arrows,snakes,backgrounds,shapes,patterns}
\usetikzlibrary{matrix,fit,positioning,decorations.pathmorphing}
\usepackage{listings}
\lstset{
        language=c,
        keywordstyle=\color{red},
        % frame=single,
        basicstyle=\ttfamily,
        commentstyle=\color{blue},
        breakindent=0pt,
        rulesepcolor=\color{red!20!green!20!blue!20},
        rulecolor=\color{black},
        tabsize=4,
        numbersep=5pt,
        breaklines=true,
        backgroundcolor=\color{green!15},
        showstringspaces=false,
        showspaces=false,
        showtabs=false,
        extendedchars=false,
        escapeinside=``,
}

\usepackage{refcount}
\usepackage{multicol}
\newcounter{countitems}
\newcounter{nextitemizecount}
\newcommand{\setupcountitems}{%
        \stepcounter{nextitemizecount}%
        \setcounter{countitems}{0}%
        \preto\item{\stepcounter{countitems}}%
}
\makeatletter
\newcommand{\computecountitems}{%
        \edef\@currentlabel{\number\c@countitems}%
        \label{countitems@\number\numexpr\value{nextitemizecount}-1\relax}%
}
\newcommand{\nextitemizecount}{%
        \getrefnumber{countitems@\number\c@nextitemizecount}%
}
\newcommand{\previtemizecount}{%
        \getrefnumber{countitems@\number\numexpr\value{nextitemizecount}-1\relax}%
}
\makeatother    
\newenvironment{AutoMultiColItemize}{%
        \ifnumcomp{\nextitemizecount}{>}{3}{\begin{multicols}{2}}{}%
                \setupcountitems\begin{itemize}}%
                {\end{itemize}%
                \unskip\computecountitems\ifnumcomp{\previtemizecount}{>}{3}{\end{multicols}}{}}

% ######### DEFINE COLOR ###############
\definecolor{blue}{rgb}{0.0,0.0,1.0}
\definecolor{red}{rgb}{1.0,0.0,0.0}
\definecolor{purple}{rgb}{0.75, 0.0, 1.0}

\def\blue{\textcolor{blue}}
\def\red{\textcolor{red}}
\def\purple{\textcolor{purple}}
\def\lst{\lstinline}
\def\ds{\displaystyle}
\def\cd{\cdots}
\def\dd{\ddots}
\def\vd{\vdots}
\def\id{\iddots}
\def\ft{\frametitle}
\def\diag{\mathrm{diag}}
\def\Im{\mathrm{Im~}}
\def\Ker{\mathrm{Ker~}}

\def\MA{\boldsymbol{A}}
\def\MB{\boldsymbol{B}}
\def\MC{\boldsymbol{C}}
\def\MD{\boldsymbol{D}}
\def\ME{\boldsymbol{E}}
\def\MF{\boldsymbol{F}}
\def\MG{\boldsymbol{G}}
\def\MH{\boldsymbol{H}}
\def\MI{\boldsymbol{I}}
\def\MP{\boldsymbol{P}}
\def\MQ{\boldsymbol{Q}}
\def\MR{\boldsymbol{R}}
\def\MS{\boldsymbol{S}}
\def\MT{\boldsymbol{T}}
\def\MU{\boldsymbol{U}}
\def\MX{\boldsymbol{X}}
\def\MY{\boldsymbol{Y}}
\def\MZ{\boldsymbol{Z}}
\def\M0{\boldsymbol{0}}
\def\MLambda{\boldsymbol{\Lambda}}

\def\R{\mathbb R}
\def\C{\mathbb C}
\def\dim{\mathrm{dim~}}
\def\rank{\mathrm{r}}
\def\tr{\mathrm{tr}}
\def\det{\mathrm{det}}
\def\vn{{\boldsymbol{n}}}
\def\vx{{\boldsymbol{x}}}
\def\vy{{\boldsymbol{y}}}
\def\vz{{\boldsymbol{z}}}
\def\va{{\boldsymbol{a}}}
\def\vb{{\boldsymbol{b}}}
\def\ve{{\boldsymbol{e}}}
\def\vi{{\boldsymbol{i}}}
\def\vj{{\boldsymbol{j}}}
\def\vk{{\boldsymbol{k}}}
\def\vu{{\boldsymbol{u}}}
\def\vv{{\boldsymbol{v}}}

\def\tf{\ttfamily}
\def\Lambdabd{\boldsymbol{\Lambda}}
\def\alphabd{\boldsymbol{\alpha}}
\def\betabd{\boldsymbol{\beta}}
\def\gammabd{\boldsymbol{\gamma}}
\def\xibd{\boldsymbol{\xi}}
\def\zetabd{\boldsymbol{\zeta}}
\def\etabd{\boldsymbol{\eta}}
\def\epsilonbd{\boldsymbol{\epsilon}}
\def\phibd{\boldsymbol{\phi}}
\def\varphibd{\boldsymbol{\varphi}}
\def\sigmabd{\boldsymbol{\sigma}}
\def\omegabd{\boldsymbol{\omega}}
\def\taubd{\boldsymbol{\tau}}
%\def\rank{\boldsymbol{rank}}

% %%%%%%%%%%%%%%%%%%%%%%%%%%%%%% 
% % Theorem
% \newcounter{theo}[section] \setcounter{theo}{0}
% \renewcommand{\thetheo}{\arabic{section}.\arabic{theo}}
% \newenvironment{theo}[2][]{%
%   \refstepcounter{theo}%
%   \ifstrempty{#1}%
%   {\mdfsetup{%
%       frametitle={%
%         \tikz[baseline=(current bounding box.east),outer sep=0pt]
%         \node[anchor=east,rectangle,fill=blue!20]
%         {\strut Theorem~\thetheo};}}
%   }%
%   {\mdfsetup{%
%       frametitle={%
%         \tikz[baseline=(current bounding box.east),outer sep=0pt]
%         \node[anchor=east,rectangle,fill=blue!20]
%         {\strut Theorem~\thetheo:~#1};}}%
%   }%
%   \mdfsetup{innertopmargin=-8pt,linecolor=blue!20,%
%     linewidth=2pt,topline=true,%
%     frametitleaboveskip=\dimexpr-\ht\strutbox\relax
%   }
%   \begin{mdframed}[]\relax%
%     \label{#2}}{\end{mdframed}}

%%%%%%%%%%%%%%%%%%%%%%%%%%%%%% 
% Question
\newcounter{question}[section] \setcounter{question}{0}
\renewcommand{\thequestion}{}%{\arabic{section}.\arabic{question}}
\newenvironment{question}[2][]{%
  \refstepcounter{question}%
  \ifstrempty{#1}%
  {\mdfsetup{%
      frametitle={%
        \tikz[baseline=(current bounding box.east),outer sep=0pt]
        \node[anchor=east,rectangle,fill=blue!20]
        {\strut 问题~\thequestion};}}
  }%
  {\mdfsetup{%
      frametitle={%
        \tikz[baseline=(current bounding box.east),outer sep=0pt]
        \node[anchor=east,rectangle,fill=blue!20]
        {\strut 问题~\thequestion:~#1};}}%
  }%
  \mdfsetup{innertopmargin=-8pt,linecolor=blue!20,%
    linewidth=2pt,topline=true,%
    frametitleaboveskip=\dimexpr-\ht\strutbox\relax
  }
  \begin{mdframed}[]\relax%
    \label{#2}}{\end{mdframed}}


% %%%%%%%%%%%%%%%%%%%%%%%%%%%%%% 
% % Property
% \newcounter{prop}[section] \setcounter{prop}{0}
% \renewcommand{\theprop}{}%\arabic{section}.\arabic{prop}}
% \newenvironment{prop}[2][]{%
%   \refstepcounter{prop}%
%   \ifstrempty{#1}%
%   {\mdfsetup{%
%       frametitle={%
%         \tikz[baseline=(current bounding box.east),outer sep=0pt]
%         \node[anchor=east,rectangle,fill=green!20]
%         {\strut Property~\theprop};}}
%   }%
%   {\mdfsetup{%
%       frametitle={%
%         \tikz[baseline=(current bounding box.east),outer sep=0pt]
%         \node[anchor=east,rectangle,fill=green!20]
%         {\strut Property~\theprop:~#1};}}%
%   }%
%   \mdfsetup{innertopmargin=-8pt,linecolor=green!20,%
%     linewidth=2pt,topline=true,%
%     frametitleaboveskip=\dimexpr-\ht\strutbox\relax
%   }
%   \begin{mdframed}[]\relax%
%     \label{#2}}{\end{mdframed}}

% %%%%%%%%%%%%%%%%%%%%%%%%%%%%%% 
% % Proposition
% \newcounter{proposition}[section] \setcounter{proposition}{0}
% \renewcommand{\theproposition}{}%{\arabic{section}.\arabic{proposition}}
% \newenvironment{proposition}[2][]{%
%   \refstepcounter{proposition}%
%   \ifstrempty{#1}%
%   {\mdfsetup{%
%       frametitle={%
%         \tikz[baseline=(current bounding box.east),outer sep=0pt]
%         \node[anchor=east,rectangle,fill=green!20]
%         {\strut Proposition~\theproposition};}}
%   }%
%   {\mdfsetup{%
%       frametitle={%
%         \tikz[baseline=(current bounding box.east),outer sep=0pt]
%         \node[anchor=east,rectangle,fill=green!20]
%         {\strut Proposition~\theproposition:~#1};}}%
%   }%
%   \mdfsetup{innertopmargin=-8pt,linecolor=green!20,%
%     linewidth=2pt,topline=true,%
%     frametitleaboveskip=\dimexpr-\ht\strutbox\relax
%   }
%   \begin{mdframed}[]\relax%
%     \label{#2}}{\end{mdframed}}

% % %%%%%%%%%%%%%%%%%%%%%%%%%%%%%% 
% % % note
% % \newcounter{note}[section] \setcounter{note}{0}
% % \renewcommand{\thenote}{}%{\arabic{section}.\arabic{note}}
% % \newenvironment{note}[2][]{%
% %   \refstepcounter{note}%
% %   \ifstrempty{#1}%
% %   {\mdfsetup{%
% %       frametitle={%
% %         \tikz[baseline=(current bounding box.east),outer sep=0pt]
% %         \node[anchor=east,rectangle,fill=green!20]
% %         {\strut Note~\thenote};}}
% %   }%
% %   {\mdfsetup{%
% %       frametitle={%
% %         \tikz[baseline=(current bounding box.east),outer sep=0pt]
% %         \node[anchor=east,rectangle,fill=green!20]
% %         {\strut Proposition~\theproposition:~#1};}}%
% %   }%
% %   \mdfsetup{innertopmargin=-8pt,linecolor=green!20,%
% %     linewidth=2pt,topline=true,%
% %     frametitleaboveskip=\dimexpr-\ht\strutbox\relax
% %   }
% %   \begin{mdframed}[]\relax%
% %     \label{#2}}{\end{mdframed}}


% %%%%%%%%%%%%%%%%%%%%%%%%%%%%%% 
% % Corallary
% \newcounter{cor}[section] \setcounter{cor}{0}
% \renewcommand{\thecor}{\arabic{section}.\arabic{cor}}
% \newenvironment{cor}[2][]{%
%   \refstepcounter{cor}%
%   \ifstrempty{#1}%
%   {\mdfsetup{%
%       frametitle={%
%         \tikz[baseline=(current bounding box.east),outer sep=0pt]
%         \node[anchor=east,rectangle,fill=green!20]
%         {\strut Corollary~\thecor};}}
%   }%
%   {\mdfsetup{%
%       frametitle={%
%         \tikz[baseline=(current bounding box.east),outer sep=0pt]
%         \node[anchor=east,rectangle,fill=green!20]
%         {\strut Corollary~\thecor:~#1};}}%
%   }%
%   \mdfsetup{innertopmargin=-8pt,linecolor=green!20,%
%     linewidth=2pt,topline=true,%
%     frametitleaboveskip=\dimexpr-\ht\strutbox\relax
%   }
%   \begin{mdframed}[]\relax%
%     \label{#2}}{\end{mdframed}}

%%%%%%%%%%%%%%%%%%%%%%%%%%%%%%

% Example
\newcounter{exam}[section] \setcounter{exam}{0}
\renewcommand{\theexam}{}%{\arabic{section}.\arabic{exam}}
\newenvironment{exam}[2][]{%
  \refstepcounter{exam}%
  \ifstrempty{#1}%
  {\mdfsetup{%
      frametitle={%
        \tikz[baseline=(current bounding box.east),outer sep=0pt]
        \node[anchor=east,rectangle,fill=blue!20]
        {\strut 例~\theexam};}}
  }%
  {\mdfsetup{%
      frametitle={%
        \tikz[baseline=(current bounding box.east),outer sep=0pt]
        \node[anchor=east,rectangle,fill=blue!20!green]
        {\strut 例~\theexam:~#1};}}%
  }%
  \mdfsetup{innertopmargin=-8pt,
    linecolor=blue!20!green,%
    linewidth=2pt,topline=true,%
    frametitleaboveskip=\dimexpr-\ht\strutbox\relax
  }
  \begin{mdframed}[]\relax%
    \label{#2}}{\end{mdframed}}


%%%%%%%%%%%%%%%%%%%%%%%%%%%%%%
% Definition
\newcounter{defn}[section] \setcounter{defn}{0}
\renewcommand{\thedefn}{}%{\arabic{section}.\arabic{defn}}
\newenvironment{defn}[2][]{%
  \refstepcounter{defn}%
  \ifstrempty{#1}%
  {\mdfsetup{%
      frametitle={%
        \tikz[baseline=(current bounding box.east),outer sep=0pt]
        \node[anchor=east,rectangle,fill=blue!20]
        {\strut 定义~\thedefn};}}
  }%
  {\mdfsetup{%
      frametitle={%
        \tikz[baseline=(current bounding box.east),outer sep=0pt]
        \node[anchor=east,rectangle,fill=blue!20!green]
        {\strut 定义~\thedefn:~#1};}}%
  }%
  \mdfsetup{innertopmargin=-8pt,linecolor=blue!20!green,%
    linewidth=2pt,topline=true,%
    frametitleaboveskip=\dimexpr-\ht\strutbox\relax
  }
  \begin{mdframed}[]\relax%
    \label{#2}}{\end{mdframed}}


% %%%%%%%%%%%%%%%%%%%%%%%%%%%%%% 
% % Lemma
% \newcounter{lem}[section] \setcounter{lem}{0}
% \renewcommand{\thelem}{\arabic{section}.\arabic{lem}}
% \newenvironment{lem}[2][]{%
%   \refstepcounter{lem}%
%   \ifstrempty{#1}%
%   {\mdfsetup{%
%       frametitle={%
%         \tikz[baseline=(current bounding box.east),outer sep=0pt]
%         \node[anchor=east,rectangle,fill=green!20]
%         {\strut Lemma~\thelem};}}
%   }%
%   {\mdfsetup{%
%       frametitle={%
%         \tikz[baseline=(current bounding box.east),outer sep=0pt]
%         \node[anchor=east,rectangle,fill=green!20]
%         {\strut Lemma~\thelem:~#1};}}%
%   }%
%   \mdfsetup{innertopmargin=-8pt,linecolor=green!20,%
%     linewidth=2pt,topline=true,%
%     frametitleaboveskip=\dimexpr-\ht\strutbox\relax
%   }
%   \begin{mdframed}[]\relax%
%     \label{#2}}{\end{mdframed}}


%%%%%%%%%%%%%%%%%%%%%%%%%%%%%% 
% 
\newcounter{free}[section]\setcounter{free}{0}
% \renewcommand{\thefree}{\arabic{section}.\arabic{free}}
\newenvironment{free}[2][]{%
  \refstepcounter{free}%
  \ifstrempty{#1}%
  {\mdfsetup{%
      frametitle={%
        \tikz[baseline=(current bounding box.east),outer sep=0pt]
        \node[anchor=east,rectangle,fill=red!20]
        {\strut ~};}}
  }%
  {\mdfsetup{%
      frametitle={%
        \tikz[baseline=(current bounding box.east),outer sep=0pt]
        \node[anchor=east,rectangle,fill=red!20]
        {\strut #1~};}}%
  }%
  \mdfsetup{innertopmargin=-8pt,linecolor=red!20,%
    linewidth=2pt,topline=true,%
    frametitleaboveskip=\dimexpr-\ht\strutbox\relax
  }
  \begin{mdframed}[]\relax%
    \label{#2}}{\end{mdframed}}
%%%%%%%%%%%%%%%%%%%%%%%%%%%%%% 


% %%%%%%%%%%%%%%%%%%%%%%%%%%%%%% 
% % Solution
% \newcounter{prf}[section]\setcounter{prf}{0}
% % \renewcommand{\theprf}{\arabic{section}.\arabic{prf}}
% \newenvironment{prf}[2][]{%
%   \refstepcounter{prf}%
%   \ifstrempty{#1}%
%   {\mdfsetup{%
%       frametitle={%
%         \tikz[baseline=(current bounding box.east),outer sep=0pt]
%         \node[anchor=east,rectangle,fill=red!20]
%         {\strut Proof~};}}
%   }%
%   {\mdfsetup{%
%       frametitle={%
%         \tikz[baseline=(current bounding box.east),outer sep=0pt]
%         \node[anchor=east,rectangle,fill=red!20]
%         {\strut Proof(#1)~:};}}%
%   }%
%   \mdfsetup{innertopmargin=-8pt,linecolor=red!20,%
%     linewidth=2pt,topline=true,%
%     frametitleaboveskip=\dimexpr-\ht\strutbox\relax
%   }
%   \begin{mdframed}[]\relax%
%     \label{#2}}{\end{mdframed}}
% %%%%%%%%%%%%%%%%%%%%%%%%%%%%%% 

% % %%%%%%%%%%%%%%%%%%%%%%%%%%%%%% 
% % % Defns


% Changing font size of selected slides in beamer
% You can use \fontsize:
% \fontsize{<font size>}{<value for \baselineskip>}\selectfont
% For example,
% \fontsize{6pt}{7.2}\selectfont
% changes the font size to 6 points and the \baselineskip to 7.2 points. 
\newcommand\Fontvi{\fontsize{6.5}{7.2}\selectfont}

% https://stackoverflow.com/questions/26878002/beamer-second-screen-with-xelatex
\renewcommand\pgfsetupphysicalpagesizes{%
  \pdfpagewidth\pgfphysicalwidth\pdfpageheight\pgfphysicalheight%
}
\makeatletter
\def\beamer@framenotesbegin{% at beginning of slide
\usebeamercolor[fg]{normal text}
\gdef\beamer@noteitems{}%
\gdef\beamer@notes{}%
}
\makeatother

\title{Data structure and algorithm in Python}
\subtitle{Linked Lists}
\date{}%{\today}
\author{Xiaoping Zhang}
\institute{School of Mathematics and Statistics, Wuhan University}
% \titlegraphic{\hfill\includegraphics[height=1.5cm]{logo.pdf}}

\begin{document}

\maketitle

\begin{frame}{Table of contents}
  %\Fontvi
  \setbeamertemplate{section in toc}[sections numbered]
  \tableofcontents[hideallsubsections]
\end{frame}

\begin{frame}
  Python’s list class is highly optimized, and often a great choice for storage. With that said, there are some notable disadvantages:
  \begin{enumerate}
  \item The length of a dynamic array might be longer than the actual number of elements that it stores.
  \item Amortized bounds for operations may be unacceptable in real-time systems.
  \item Insertions and deletions at interior positions of an array are expensive.
  \end{enumerate}

  In this lecture, we introduce a data structure known as a \red{linked list}, which provides an alternative to an \blue{array-based sequence} (such as a Python list).
\end{frame}

\begin{frame}
   Both array-based sequences and linked lists keep elements in a certain order, but using a very different style.
  \begin{itemize}
  \item An \blue{array} provides the more centralized representation, with one large chunk of memory capable of accommodating references to many elements.
  \item A \blue{linked list}, in contrast, relies on a more distributed representation in which a lightweight object, known as a \red{node}, is allocated for each element.
  \item[] Each node maintains a reference to its element and one or more references to neighboring nodes in order to collectively represent the linear order of the sequence.
  \end{itemize}
\end{frame}

\section{简单实例}
%\lstset{frameshape={RYRYNYYYY}{yny}{yny}{RYRYNYYYY}}

\begin{frame}{\subsecname}
  % \lstinputlisting[
  %   language=c,
  %   frame=single,
  %   numbers=left
  % ]{Code/first.c}
\end{frame}








\section{while语句}

\begin{frame}[fragile]\ft{\secname}
\begin{lstlisting}[language=c,backgroundcolor=\color{red!10}]
while (condition)
  statement
\end{lstlisting}
\begin{lstlisting}[language=c,backgroundcolor=\color{red!10}]
while (condition)
{
  statements
}
\end{lstlisting}
\end{frame}

\begin{frame}[fragile]\ft{\secname}
\begin{figure}
\centering
\tikzstyle{startstop}=[rectangle,rounded corners,minimum width=3cm,minimum height=1cm,text centered,draw=black,fill=red!30]
\tikzstyle{process}=[rectangle,minimum width=3cm,minimum height=1cm,text centered,draw=black,fill=orange!30]
\tikzstyle{decision}=[diamond,aspect=2,minimum width=3cm,minimum height=.5cm,text centered,draw=black,fill=green!30]
\tikzstyle{arrow}=[thick,->,>=stealth]

\begin{tikzpicture}[node distance=2.5cm]
\node (start) [] {};
\node (dec) [decision,below of=start] {condition};
\node (pro) [process,below of=dec,align=left] 
{\{\\
~~~~~statements;\\
\}
};
\node (right) [right of=pro] {};

\draw [arrow] (start) -- (dec);
\draw [arrow] (dec) -- node[anchor=east] {yes} (pro);
\draw [thick] (pro) -- ++(4,0);
\draw [arrow] (pro)++(4,0)|- (dec);
\draw [arrow] (dec) -- node[anchor=south] {no}
node[align=left,anchor=north] {go to next  \\ statement} ++(-4,0);
\end{tikzpicture}

\end{figure}

\end{frame}

\begin{frame}[fragile]\ft{\secname:终止while循环}
构造一个while循环时,必须能改变判断表达式的值,并最终使其为假,否则循环永远不会终止。
\end{frame}

\begin{frame}[fragile]\ft{\secname:终止while循环}
\begin{lstlisting}[language=c]
index = 1;
while (index < 5)
{
  printf("Good morning!\n");
}
\end{lstlisting} 
\rule{\textwidth}{1mm}\pause \vspace{0.1in}

这一段代码无法终止循环,因为在循环中不能改变index的值。
\end{frame}

\begin{frame}[fragile]\ft{\secname:终止while循环}
\begin{lstlisting}[language=c]
index = 1;
while (--index < 5)
{
  printf("Good morning!\n");
}
\end{lstlisting} 
\rule{\textwidth}{1mm}\pause \vspace{0.1in}

虽然改变了index的值,但却朝着错误的方向,故仍无法退出循环。
\end{frame}

\begin{frame}[fragile]\ft{\secname:终止while循环}
\begin{lstlisting}[language=c]
index = 1;
while (++index < 5)
{
  printf("Good morning!\n");
}
\end{lstlisting} 
\rule{\textwidth}{1mm}\pause \vspace{0.1in}

这段代码可以正常退出循环。
\end{frame}

\begin{frame}[fragile]\ft{\secname:何时终止循环}
只有在计算判断条件的值时才能决定是否终止循环。
\end{frame}

\begin{frame}[fragile]\ft{\secname:何时终止循环}
  \begin{minipage}{0.65\textwidth}
\lstinputlisting[language=c,frame=single,numbers=left]{Code/when.c}    
  \end{minipage}~~~~
  \begin{minipage}{0.3\textwidth}
\begin{lstlisting}[backgroundcolor=\color{red!10}]
n = 5
Now n = 6
n = 6
Now n = 7
\end{lstlisting}
    
  \end{minipage}


\end{frame}

\begin{frame}[fragile]\ft{\secname:while:入口条件循环}
while循环是使用入口条件的有条件循环。
\end{frame}

\begin{frame}[fragile]\ft{\secname:何时终止循环}
\begin{lstlisting}[language=c]
index = 10;
while (index++ < 5)
  printf("Have a fair day or better.\n");
\end{lstlisting}
\rule{\textwidth}{1mm}\pause 

把第一行改为
\begin{lstlisting}
index = 3;
\end{lstlisting}
就可以执行这个循环了。
\end{frame}

\begin{frame}[fragile]\ft{\secname:语法要点}
在使用while时,请确定循环体的范围。缩进是为了帮助读者而不是计算机。
\end{frame}

\begin{frame}[fragile]\ft{\secname:语法要点}
  \begin{minipage}{0.6\textwidth}
\lstinputlisting[language=c,frame=single,numbers=left]{Code/while1.c}    
  \end{minipage} ~~~~\pause 
  \begin{minipage}{0.3\textwidth}
\begin{lstlisting}[backgroundcolor=\color{red!10}]
n = 0
n = 0
n = 0
n = 0
... 
\end{lstlisting}    
  \end{minipage}
\end{frame}

\begin{frame}[fragile]\ft{\secname:语法要点}
while语句在语法上算作一条单独的语句,即使它使用了复合语句。
该语句从while开始,到第一个分号结束;在使用了复合语句的情况下,到终结花括号结束。
\end{frame}

\begin{frame}[fragile]\ft{\secname:语法要点}
\lstinputlisting[language=c,frame=single,numbers=left]{Code/while2.c}
\pause 
\begin{lstlisting}[backgroundcolor=\color{red!10}]
n = 4
That's all this program does.
\end{lstlisting}
\end{frame}

\begin{frame}[fragile]\ft{\secname:语法要点}
在C语言中,\textcolor{acolor1}{单独的分号代表空语句(null statement)。}
\end{frame}

\begin{frame}[fragile]\ft{\secname:语法要点}
有些时候,程序员会有意地使用带空语句的while语句。例如,假定你想要跳过输入直到第一个不为空或数字的字符,可以这样做。
\rule{\textwidth}{1mm}\pause 

\begin{lstlisting}[language=c]
while(scanf("%d",&num)==1)
  ;
\end{lstlisting}\pause

请注意,为了清楚起见,请把分号单独置于while的下一行。
\end{frame}


\section{使程序可读的技巧}
\begin{frame}[fragile]\ft{提高程序可读性}
\begin{itemize}
\item 变量命名时做到“见其名知其意”;\\[0.1in]
\item 合理使用注释;\\[0.1in]
\item 使用空行分隔一个函数的各个部分,如声明、操作等;\\[0.1in]
\item 每条语句用一行。注意,C允许把多条语句放在同一行或一条语句放多行;\\[0.1in]
\item 建议在程序开始处用一个注释说明文件名和程序的作用。该过程花不了多少时间,但对以后浏览或打印程序很有帮助;\\[.1in]
\item 当程序比较复杂时,使用多个函数会可实现程序的模块化,使程序可读性更强。
\end{itemize}
\end{frame}


% \begin{frame}[fragile]\ft{提高程序可读性}
% \begin{lstlisting}[
% language=c,
% frame=single,
% numbers=left
% ]
% // mile_km.c: Convert 2 miles to kilometers
% #include<stdio.h>
% int main(void) 
% {
%   float mile, km;       
%   mile = 2;
%   km = 1.6 * mile;
%   printf("%d mile = %d km\n", mile, km);
%   printf("Yes, %d km\n", 1.6 * mile);
%   return 0;
% }
% \end{lstlisting}
% \end{frame}


% \begin{frame}[fragile]{\secname}

% \begin{lstlisting}[
% language=c,
% frame=single,
% % numbers=left
% ]
%   float mile, km;
% \end{lstlisting}
% 等同于
% \begin{lstlisting}[
% language=c,
% frame=single,
% % numbers=left
% ]
%   float mile;
%   float km;
% \end{lstlisting}
% \end{frame}

% \begin{frame}[fragile]\ft{\secname}

%   \begin{itemize}
%   \item 第一个 \lstinline|printf| 语句用了两个占位符:
%     \begin{itemize}
%     \item 第一个 \lstinline|%d| 为  \lstinline|mile| 占位
%     \item 第二个 \lstinline|%d| 为  \lstinline|km| 占位
%     \item 圆括号 \lstinline|()| 中有三个参数,之间用逗号隔开。
%     \end{itemize}

%   \item  第二个 \lstinline|printf| 语句说明输出的值可以是一个表达式。
%   \end{itemize}


% \end{frame}

\subsection{重定向与文件}



\begin{frame}[fragile]\ft{\subsecname}
\end{frame}
\section{条件运算符}
\begin{frame}[fragile]\ft{\secname}
C提供一种简写方式来表示{\tf if else}语句,被称为条件表达式,并使用条件运算符{(\tf ? :)}。它是C语言中唯一的三目操作符。
\end{frame}

\begin{frame}[fragile]\ft{\secname}
\begin{lstlisting}[title=求绝对值]
x = (y < 0) ? -y : y;
\end{lstlisting}
\rule{\textwidth}{.8mm} \pause \vspace{.05in}

\begin{itemize}
\item
含义:若y小于0,则x = -y;否则,x = y。\\[0.1in]
\item 用if else描述为
\begin{lstlisting}
if (y < 0)
  x = -y;
else
  x = y;  
\end{lstlisting}
\end{itemize}
\end{frame}

\begin{frame}[fragile]\ft{\secname}
\begin{lstlisting}[title=条件表达式的语法]
expresion1 ? expression2 : expression3
\end{lstlisting}
\rule{\textwidth}{.8mm} \pause \vspace{.05in}

若expresion1为真,则条件表达式的值等于expression2的值;
若expresion1为假,则条件表达式的值等于expression3的值。
\end{frame}

\begin{frame}[fragile]\ft{\secname}
若希望将两个可能的值中的一个赋给变量时,可使用条件表达式。典型的例子是将两个值中的最大值赋给变量:
\begin{lstlisting}[frame=single]
max = (a > b) ? a : b;
\end{lstlisting}
\end{frame}

\begin{frame}[fragile]\ft{\secname}
if else语句能完成与条件运算符同样的功能。但是,条件运算符语句更简洁;并且可以产生更精简的程序代码。
\end{frame}

\begin{frame}[fragile]\ft{\secname}
\begin{li}
设每罐油漆可喷200平方英尺,编写程序计算向给定的面积喷油漆,全部喷完需要多少罐油漆。
\end{li}
\end{frame}

\begin{frame}[fragile]\ft{\secname}
\lstinputlisting[language=c,numbers=left,frame=single]{slide07/code/paint.c}

\end{frame}

\begin{frame}[fragile]\ft{\secname}
\begin{lstlisting}[backgroundcolor=\color{blue!20}]
Enter number of square feet to be painted:
200
You need 1 can of paint.
Enter next value (q to quit):
225
You need 2 cans of paint.
Enter next value (q to quit):
q
\end{lstlisting}
\end{frame}



\section{改变调用函数中的变量}

\begin{frame}[fragile]\ft{\secname}
有些时候,我们需要用一个函数改变另一个函数的变量。如排序问题中,一个常见的任务是交换两个变量的值。
\end{frame}

\begin{frame}[fragile]\ft{\secname}
以下代码能否交换变量 \lstinline|x| 和 \lstinline|y| 的值{\Large ?}
\begin{lstlisting}[backgroundcolor=\color{red!10}]
x = y;
y = x;
\end{lstlisting}
\pause \vspace{0.1in}

\begin{center}
{\Large NO!}
\end{center}
\pause\vspace{0.1in}

\begin{center}
{\Large Why?} 
\end{center}
\end{frame}

\begin{frame}[fragile]\ft{\secname}
那以下代码能否交换变量 \lstinline|x| 和 \lstinline|y| 的值{\Large ?}
\begin{lstlisting}[backgroundcolor=\color{red!10}]
temp = y;
x = y;
y = temp;
\end{lstlisting}
\pause \vspace{0.1in}

\begin{center}
{\Large OK!}
\end{center}

\end{frame}

\begin{frame}[fragile,allowframebreaks]\ft{\secname}
  \lstinputlisting
  []
  {slide09/code/swap1.c}
\end{frame}


\begin{frame}[fragile]\ft{\secname}
\begin{lstlisting}[backgroundcolor=\color{red!10}]
Originally: x = 5, y = 10.
Now       : x = 5, y = 10.
\end{lstlisting}
\pause \vspace{0.1in}

\begin{center}
{\Large Why not interchanged?}
\end{center}


\end{frame}

\begin{frame}[fragile,allowframebreaks]\ft{\secname}
  \lstinputlisting
  []
  {slide09/code/swap2.c}
\end{frame}

\begin{frame}[fragile]\ft{\secname}
\begin{lstlisting}[backgroundcolor=\color{red!10}]
Originally: x =  5, y = 10.
Originally: u =  5, v = 10.
Now       : u = 10, v =  5.
Now       : x =  5, y = 10.
\end{lstlisting}
\pause \vspace{0.1in}

\begin{itemize}
\item
在 \lstinline|interchange()| 中, \lstinline|u| 和 \lstinline|v| 的值确实得到了交换。问题出在了把执行结果传递给函数main的时候。\\[0.1in]
\item
  \lstinline|interchange()| 中的变量独立于 \lstinline|main()| ,因此交换 \lstinline|u| 和 \lstinline|v| 的值对 \lstinline|x| 和 \lstinline|y| 的值没有任何影响。
\end{itemize}

\end{frame}

\begin{frame}[fragile]\ft{\secname}
能否使用 \lstinline|return|?如
\begin{lstlisting}[backgroundcolor=\color{red!10}]
int main(void)
{
  ...
  x = interchange(x, y);
  ...
}
int interchange(int u, int v)
{
  int temp;
  temp = u;
  u = v;
  v = temp;
  return u;
}
\end{lstlisting}
\end{frame}

\begin{frame}[fragile]\ft{\secname}
此时, \lstinline|x| 的值得以更新,但 \lstinline|y| 的值仍未做改变。因为\red{ \lstinline|return| 语句只能把一个数值传递给调用函数},而现在却需要传递两个数值。
\pause \vspace{0.1in}

\begin{center}
{\Large 怎么办?}
\end{center}
\pause\vspace{0.1in}

\begin{center}
{\Large 用指针!} 
\end{center}
\end{frame}



\end{document}
