\section{goto 语句}

\begin{frame}[fragile]\ft{\secname}
\begin{lstlisting}[language=c,frame=single]
// goto`语法`
goto label;
...
label: printf("Refined analysis.\n");
\end{lstlisting}

\begin{itemize}
\item
goto语句包括两个部分:goto和一个标签。标签的命名方式与变量命名相同。\\[0.1in]
\item 
必须包含由标签定位的其它语句:标签名后紧跟一个冒号,然后是一条语句。
\end{itemize}
\end{frame}

\begin{frame}[fragile]\ft{\secname}
goto语句非常容易被滥用,建议谨慎使用,或者根本不用。
\end{frame}

\begin{frame}[fragile]\ft{\secname}
\begin{lstlisting}[language=c,frame=single]
if (size > 12)
  goto a;
goto b;
a: cost *= 1.05;
  flag = 2;
b: bill = cost * flag;    
\end{lstlisting}
等效于
\begin{lstlisting}[language=c,frame=single]
if (size > 12)
{
  cost *= 1.05;
  flag = 2;
}  
bill = cost * flag;    
\end{lstlisting}
\end{frame}

\begin{frame}[fragile]\ft{\secname}
\begin{lstlisting}[language=c,frame=single]
if (n > 14)
  goto a;
m = 2;  
goto b;
a: m = 3;
b: k = 2 * m;    
\end{lstlisting}
等效于
\begin{lstlisting}[language=c,frame=single]
if (size > 12)
  m = 2;
else
  n = 3;   
k = 2 * m;    
\end{lstlisting}
\end{frame}


\begin{frame}[fragile]\ft{\secname}
\begin{lstlisting}[language=c,frame=single]
readin: scanf("%d", &score);
if (score < 0)
  goto stage2;
lots of statements;
goto readin;
stage2: morestuff;      
\end{lstlisting}
等效于
\begin{lstlisting}[language=c,frame=single]
scanf("%d", &score); 
while (score >= 0)
{
  lots of statements;
  scanf("%d", &score);
}   
more stuff;
\end{lstlisting}
\end{frame}


\begin{frame}[fragile]\ft{\secname}
\begin{itemize}
\item 调到循环末尾并开始下一轮循环,用continue代替。\\[0.1in]
\item 跳出循环,用break代替。\\[0.1in]
\item 事实上,break和continue是goto的特殊形式。使用它们的好处是其名称表明会完成什么动作;并且不需要标签,故不存在错放标签位置的潜在危险。
\end{itemize}
\end{frame}
