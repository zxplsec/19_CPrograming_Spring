\section{上机操作}

\begin{frame}[fragile]
\begin{free}[例]{}
编写一个程序,读取输入直到遇到字符 \lstinline|#|,然后报告读取的空格数目、换行符数目以及其它字符数目。
\end{free}
\end{frame}

\begin{frame}[fragile,allowframebreaks]
\lstinputlisting[]{slide07/code_op/ex01.c}
\end{frame}

\begin{frame}[fragile]
\begin{free}[例]{}
编写一个程序,读取输入直到遇到字符 \lstinline|#|,使程序打印每个输入的字符以及它的十进制ASCII码,每行打印8个字符。
\end{free}
\end{frame}

\begin{frame}[fragile,allowframebreaks]
\lstinputlisting[]{slide07/code_op/ex02.c}
\end{frame}

\begin{frame}[fragile]
\begin{free}[例]{}
编写一个程序,读取整数直到输入0。输入终止后,程序应该报告输入的偶数(不包括0)总个数及其平均值,奇数总个数及其平均值。
\end{free}
\end{frame}

\begin{frame}[fragile,allowframebreaks]
\lstinputlisting[]{slide07/code_op/ex03.c}
\end{frame}


\begin{frame}[fragile]
\begin{free}[例]{}
利用if else语句编写程序读取输入,直到 \lstinline|#|。用一个感叹号代替每一个句号,将原有的每个感叹号用两个感叹号代替,最后报告进行了多少次替代。
\end{free}
\end{frame}

\begin{frame}[fragile,allowframebreaks]
\lstinputlisting[]{slide07/code_op/ex04.c}
\end{frame}


% \begin{frame}[fragile]
% \begin{free}[例]{}
% 编写一个程序读取输入,直到 \lstinline|#|,并报告序列ei出现的次数。
% \end{free}
% \end{frame}

\end{document}
