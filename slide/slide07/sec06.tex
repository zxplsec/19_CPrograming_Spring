\section{ \lstinline|continue| 和 \lstinline|break| 语句}
\begin{frame}[fragile]\ft{\secname}
 \lstinline|continue| 和 \lstinline|break| 语句用于循环结构,根据判断条件来忽略部分循环甚至终止循环。
\end{frame}


\begin{frame}[fragile]\ft{\secname: \lstinline|continue| 语句}
\begin{itemize}
\item
当程序运行到 \lstinline|continue| 语句时,其后的内容将被忽略,开始进入下一次循环。\\[0.1in]
\item
当 \lstinline|continue| 语句用于嵌套结构时,仅影响包含它的那一层循环。
\end{itemize}
\end{frame}


\begin{frame}[fragile]\ft{\secname: \lstinline|continue| 语句}
\begin{free}[例]{}
输入1-100之间的多个分数,求其平均分、最低分和最高分。当输入分数不在1-100之间时,程序应该不做处理。
\end{free}

\end{frame}


\begin{frame}[fragile,allowframebreaks]\ft{\secname: \lstinline|continue| 语句}
\lstinputlisting
{slide07/code/continue.c}
\end{frame}

\begin{frame}[fragile]\ft{\secname: \lstinline|continue| 语句}
\begin{lstlisting}[backgroundcolor=\color{red!10}]
Enter the first score (q to quit): 20
Accepting 20.0:
Enter next score (q to quit): -1
-1.0 is invalid. Try again: 90
Accepting 90.0:
Enter next score (q to quit): 110
110.0 is invalid. Try again: q
Average of 2 scores is 55.0.
Low = 20.0, High = 90.0.
\end{lstlisting}

\end{frame}

\begin{frame}[fragile]\ft{\secname: \lstinline|continue| 语句}
\begin{itemize}
\item 对于 \lstinline|while| 和 \lstinline|do while| 循环, \lstinline|continue| 语句之后发生的动作是求循环表达式的值。\\[0.1in]
\item 而对于 \lstinline|for| 循环,下一个动作是先求更新表达式的值,然后再求判断表达式的值。
\end{itemize}

\end{frame}

\begin{frame}[fragile]\ft{\secname: \lstinline|continue| 语句}
\begin{lstlisting}
count = 0;
while (count < 10)
{
  ch = getchar();
  if (ch == '\n')
    continue;
  putchar(ch);
  count++;
}
\end{lstlisting}
读取除换行符外的10个字符,并回显它们。注意:换行符不会被计数。
\end{frame}

\begin{frame}[fragile]\ft{\secname: \lstinline|continue| 语句}
\begin{lstlisting}
for (count = 0; count < 10; count++)
{
  ch = getchar();
  if (ch == '\n')
    continue;
  putchar(ch);
}
\end{lstlisting}
读取包含换行符在内的10个字符,换行符不被回显,但会被计数。
\end{frame}

\begin{frame}[fragile]\ft{\secname: \lstinline|break| 语句}
\begin{itemize}
\item
当程序运行到 \lstinline|break| 语句时,将会终止包含它的循环,跳出该循环体。\\[0.1in]
\item
当 \lstinline|break| 语句用于嵌套结构时,仅影响包含它的那一层循环。
\end{itemize}
\end{frame}

\begin{frame}[fragile]\ft{\secname: \lstinline|break| 语句}
\begin{free}[例]{}
输入矩形的长和宽,用一个循环来计算其面积。若输入一个非数字作为矩形的长或宽,终止循环。
\end{free}
\end{frame}

\begin{frame}[fragile,allowframebreaks]\ft{\secname: \lstinline|break| 语句}

\lstinputlisting
            {slide07/code/break.c}
\end{frame}


\begin{frame}[fragile]\ft{\secname: \lstinline|continue| 语句}
\begin{lstlisting}[backgroundcolor=\color{red!10}]
Enter the length of the rectangle: 10
Length = 10.00.
Enter its width: 20
Width = 20.00;
Area = 200.00; 
Enter the length of the rectangle: 10
Length = 10.00.
Enter its width: q
Done.
\end{lstlisting}
\end{frame}

\begin{frame}[fragile]\ft{\secname: \lstinline|break| 语句}
\begin{itemize}
\item  \lstinline|break| 语句使程序直接跳转到该循环后的第一条语句;在for循环中,更新表达式也将被跳过。\\[0.1in]
\item 
嵌套循环中, \lstinline|break| 语句只能使程序跳出当前循环,要跳出外层循环还需另外一个 \lstinline|break| 语句。
\end{itemize}
\end{frame}


\begin{frame}[fragile]\ft{\secname: \lstinline|break| 语句}
\begin{lstlisting}[language=c,frame=single]
int p, q;
scanf("%d", &p);
while (p > 0) {
  printf("%d\n", p);
  scanf("%d", &q);
  while (q > 0) {
    printf("%d\n", p*q);
    if (q > 100)
       break;
    scanf("%d", &q);
  }
  if (q > 100)
    break;
  scanf("%d", &p);
}
\end{lstlisting}
\end{frame}


