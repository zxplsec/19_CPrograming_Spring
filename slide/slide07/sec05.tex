\section{条件运算符}
\begin{frame}[fragile]\ft{\secname}
  C提供一种简写方式来表示 \lstinline|if else| 语句,被称为条件表达式,并使用条件运算符 \lstinline|? :|。它是C语言中唯一的三目操作符。
\end{frame}

\begin{frame}[fragile]\ft{\secname}
\begin{lstlisting}[title=求绝对值]
x = (y < 0) ? -y : y;
\end{lstlisting}
\rule{\textwidth}{.8mm} \pause \vspace{.05in}

\begin{itemize}
\item
  含义:若 \lstinline|y| 小于 \lstinline|0|,则 \lstinline|x = -y|;
  否则,\lstinline|x = y|。\\[0.1in]
\item 用if else描述为
\begin{lstlisting}
if (y < 0)
  x = -y;
else
  x = y;  
\end{lstlisting}
\end{itemize}
\end{frame}

\begin{frame}[fragile]\ft{\secname}
\begin{lstlisting}[title=条件表达式的语法]
expression1 ? expression2 : expression3
\end{lstlisting}
\rule{\textwidth}{.8mm} \pause \vspace{.05in}

\begin{itemize}
\item 若 \lstinline|expression1| 为真,则条件表达式的值等于 \lstinline|expression2| 的值;
\item 若 \lstinline|expression1| 为假,则条件表达式的值等于 \lstinline|expression3| 的值。
\end{itemize}
\end{frame}

\begin{frame}[fragile]\ft{\secname}
  若希望将两个可能值中的一个赋给变量时,可使用条件表达式。


  典型的例子是将两个值中的最大值赋给变量:
\begin{lstlisting}[frame=single]
max = (a > b) ? a : b;
\end{lstlisting} \pause 


\lstinline|if else| 语句能完成与条件运算符同样的功能。但是,条件运算符语句更简洁;并且可以产生更精简的程序代码。
\end{frame}

\begin{frame}[fragile]\ft{\secname}
\begin{free}[例]{}
设每罐油漆可喷200平方英尺,编写程序计算向给定的面积喷油漆,全部喷完需要多少罐油漆。
\end{free}
\end{frame}

\begin{frame}[fragile]\ft{\secname}
  \lstinputlisting
  {slide07/code/paint.c}

\end{frame}

\begin{frame}[fragile]\ft{\secname}
\begin{lstlisting}[backgroundcolor=\color{blue!20}]
Enter number of square feet to be painted:
200
You need 1 can of paint.
Enter next value (q to quit):
225
You need 2 cans of paint.
Enter next value (q to quit):
q
\end{lstlisting}
\end{frame}


