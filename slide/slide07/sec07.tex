\section{switch语句}
\begin{frame}[fragile]\ft{\secname}
多重选择时,可以使用
\begin{lstlisting}
if (condition1)
  ...
else if (condition2)
  ...
else if (condition3)
  ...
else    
\end{lstlisting}
但多数情况下,使用switch语句会更加方便。
\end{frame}

\begin{frame}[fragile,allowframebreaks]\ft{\secname}
\lstinputlisting[language=c,numbers=left,frame=single]
{slide07/code/animals.c}
\end{frame}


\begin{frame}[fragile]\ft{\secname}
\begin{lstlisting}[backgroundcolor=\color{blue!20}]
Give a letter, and I will give you an animal name beginning with that letter.
Please type in a letter: # to quit.
dog
dove
Please typer another letter or a #.
a
alligator
Please typer another letter or a #.
eff
eagle
Please typer another letter or a #.
#
Bye!
\end{lstlisting}
\end{frame}

\begin{frame}[fragile]\ft{\secname}
\begin{lstlisting}[language=c,frame=single]
// switch`语法`
switch (integer expression)
{
  case constant1:
    statements
  case constant2:
    statements
  ...
  default:
    statements
}    
\end{lstlisting}
\end{frame}

\begin{frame}[fragile]\ft{\secname}
  1、判断表达式应该具有整数值,包括int,char和enum类型。
\end{frame}

\begin{frame}[fragile]\ft{\secname}
  \lstinputlisting[language=c,numbers=left,frame=single]
  {slide07/code/switch1.c}
\end{frame}

\begin{frame}[fragile]\ft{\secname}
2、break使程序跳出switch结构,执行switch之后的下一条语句。
若没有break语句,从相匹配的标签到switch末尾的每一条语句都会被执行。
\end{frame}

\begin{frame}[fragile]\ft{\secname}
  \lstinputlisting[language=c,numbers=left,frame=single]
  {slide07/code/switch2.c}
\end{frame}

\begin{frame}[fragile]\ft{\secname}
\begin{lstlisting}[backgroundcolor=\color{blue!20}]
Choice is 2
Choice is 3
Choice other than 1, 2 and 3
\end{lstlisting}

\end{frame}

\begin{frame}[fragile]\ft{\secname}
  \lstinputlisting[language=c,numbers=left,frame=single]
  {slide07/code/switch3.c}
\end{frame}

\begin{frame}[fragile]\ft{\secname}
\begin{lstlisting}[backgroundcolor=\color{blue!20}]
Choice is 2
Choice is 3
Choice is 4
After Switch
\end{lstlisting}
\end{frame}


\begin{frame}[fragile]\ft{\secname}
3、case标签必须是整型常量或整型常量表达式,不能用变量作为case标签。
\end{frame}

\begin{frame}[fragile]\ft{\secname}
  \lstinputlisting[language=c,numbers=left,frame=single]
  {slide07/code/switch4.c}
\end{frame}

\begin{frame}[fragile]\ft{\secname}
4、default语句块可放在switch结构中的任意位置,若判断表达式与标签均不匹配,它会被执行。
\end{frame}

\begin{frame}[fragile]\ft{\secname}
  \lstinputlisting[language=c,numbers=left,frame=single]
  {slide07/code/switch5.c} \pause 
\begin{lstlisting}[backgroundcolor=\color{blue!20}]
Choice other than 1 and 2
\end{lstlisting}
\end{frame}


\begin{frame}[fragile,allowframebreaks]\ft{\secname}
5、case之前的语句不会被执行。一旦进入switch结构,将直接转入标签匹配。
\end{frame}

\begin{frame}[fragile,allowframebreaks]\ft{\secname}
\lstinputlisting
[language=c,numbers=left,frame=single]
{slide07/code/switch6.c}
\begin{lstlisting}[backgroundcolor=\color{blue!20}]
Choice is 1
\end{lstlisting}
\end{frame}

\begin{frame}[fragile,allowframebreaks]\ft{\secname}
6、两个case标签不能有相同值。
\end{frame}

\begin{frame}[fragile,allowframebreaks]\ft{\secname}
\lstinputlisting
[language=c,numbers=left,frame=single]
{slide07/code/switch7.c}
\end{frame}



\begin{frame}[fragile]\ft{\secname}
  \begin{li}
    编制程序,输入一段文字,按{\tf \#}停止输入,然后统计该段文字中字母a, e, i, o, u出现的次数(不计大小写)。
  \end{li}
\end{frame}



\begin{frame}[fragile,allowframebreaks]\ft{\secname}
\lstinputlisting
[language=c,numbers=left,frame=single]
{slide07/code/vowels.c}
\end{frame}

\begin{frame}[fragile]\ft{\secname}
\begin{lstlisting}[backgroundcolor=\color{red!10}]
Enter some text: enter # to quit.
See you tommorrow!#
Number of text:    A    E    I    U
                   0    2    0    1
\end{lstlisting}   
\end{frame}

\begin{frame}[fragile]\ft{\secname}
\begin{itemize}
\item
若输入字母为i,则switch语句定位到标签为case 'i':的位置。因没有break同该标签相关联,故程序将前进到下一条语句,即ni++。\\[0.1in]
\item 
若输入字母为I,程序将直接定位到这条语句。\\[0.1in]
\item 
本质上,两个标签都指向相同的语句。
\end{itemize} 
\end{frame}

\begin{frame}[fragile]\ft{\secname}
在该例中,可通过ctype.h中的toupper函数在进行判断之前将所有的字母转换为大写字母以避免多重标签。
\end{frame}

\begin{frame}[fragile]\ft{\secname}
\begin{lstlisting}
ch = toupper(ch);
switch (ch) {
  case 'A':  na++;
    break;
  case 'E':  ne++;
    break;
  case 'I':  ni++;
    break;
  case 'O':  no++;
    break;
  case 'U':  nu++;
    break;
  default:
    break;
}   
\end{lstlisting}           
\end{frame}

\begin{frame}[fragile]\ft{\secname}
若希望保留ch的值不变,可以这么做
\begin{lstlisting}
switch (toupper(ch)) {
  case 'A':  na++;
    break;
  case 'E':  ne++;
    break;
  case 'I':  ni++;
    break;
  case 'O':  no++;
    break;
  case 'U':  nu++;
    break;
  default:
    break;
}
\end{lstlisting}  
\end{frame}

\begin{frame}[fragile]\ft{\secname:switch与if else}
\begin{itemize}
\item
若选择是基于求一个浮点型变量或表达式的值,就不能使用switch。\\[0.1in]
\item
若变量必须落入某个范围,使用if语句会更方便。如
\begin{lstlisting}
if (integer < 1000 && integer > 2)  
\end{lstlisting}
\vspace{0.1in}
\item 若可以使用switch,程序会运行得稍快些,并且代码会更紧凑。 
\end{itemize}         
\end{frame}




