% -*- coding: utf-8 -*-
% !TEX program = xelatex

\documentclass[12pt,notheorems]{beamer}

\usetheme[style=beta]{epyt} % alpha, beta, delta, gamma, zeta

\usepackage[UTF8,noindent]{ctex}
\input{Macro/usepackage}
\usepackage{courier}
\usepackage{animate}
\usepackage{dcolumn}



\newcommand{\mylead}[1]{\textcolor{acolor1}{#1}}
\newcommand{\mybold}[1]{\textcolor{acolor2}{#1}}
\newcommand{\mywarn}[1]{\textcolor{acolor3}{#1}}

\input{Macro/newcommand}



%\newtheorem{theorem}{定理}
%\newtheorem{definition}[theorem]{定义}
%\newtheorem{example}[theorem]{例子}
%\newtheorem{dingli}[theorem]{定理}
%\newtheorem{li}[theorem]{例}
%
%\newtheorem*{theorem*}{定理}
%\newtheorem*{definition*}{定义}
%\newtheorem*{example*}{例子}
%\newtheorem*{dingli*}{定理}
%\newtheorem*{li*}{例}

\renewcommand{\proofname}{证明}
\newtheorem*{jie}{解}
\newtheorem*{zhu}{注}
\newtheorem*{dingli}{定理} 
\newtheorem*{dingyi}{定义} 
\newtheorem*{xingzhi}{性质} 
\newtheorem*{tuilun}{推论} 
\newtheorem*{li}{例} 
\newtheorem*{jielun}{结论} 
\newtheorem*{zhengming}{证明}
\newtheorem*{wenti}{问题}
\newtheorem*{jieshi}{解释}


\renewcommand{\proofname}{证明}

\begin{document}

\title{C语言}
\subtitle{第四讲、字符串与格式化输入输出}
\author{张晓平}
\institute{武汉大学数学与统计学院}


\begin{frame}[plain]\transboxout
\titlepage
\end{frame}

\begin{frame}\transboxin
\begin{center}
\tableofcontents[]%hideallsubsections]
\end{center}
\end{frame}

\AtBeginSection[]{
\begin{frame}[allowframebreaks]
\tableofcontents[currentsection,sectionstyle=show/hide]
\end{frame}
}
%\AtBeginSubsection[]{
%\begin{frame}[allowframebreaks]
%\tableofcontents[currentsection,currentsubsection,subsectionstyle=show/shaded/hide]
%\end{frame}
%}


\input{sec4.1}
\input{sec4.2}
\input{sec4.3}
\input{sec4.4}



\end{document}
