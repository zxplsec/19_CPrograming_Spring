% -*- coding: utf-8 -*-
% !TEX program = xelatex

\documentclass[12pt,notheorems]{beamer}

\usetheme[style=beta]{epyt} % alpha, beta, delta, gamma, zeta

\usepackage[UTF8,noindent]{ctex}
\input{Macro/usepackage}
\usepackage{courier}
\usepackage{animate}
\usepackage{dcolumn}



\newcommand{\mylead}[1]{\textcolor{acolor1}{#1}}
\newcommand{\mybold}[1]{\textcolor{acolor2}{#1}}
\newcommand{\mywarn}[1]{\textcolor{acolor3}{#1}}

\input{Macro/newcommand}



%\newtheorem{theorem}{定理}
%\newtheorem{definition}[theorem]{定义}
%\newtheorem{example}[theorem]{例子}
%\newtheorem{dingli}[theorem]{定理}
%\newtheorem{li}[theorem]{例}
%
%\newtheorem*{theorem*}{定理}
%\newtheorem*{definition*}{定义}
%\newtheorem*{example*}{例子}
%\newtheorem*{dingli*}{定理}
%\newtheorem*{li*}{例}

\renewcommand{\proofname}{证明}
\newtheorem*{jie}{解}
\newtheorem*{zhu}{注}
\newtheorem*{dingli}{定理} 
\newtheorem*{dingyi}{定义} 
\newtheorem*{xingzhi}{性质} 
\newtheorem*{tuilun}{推论} 
\newtheorem*{li}{例} 
\newtheorem*{jielun}{结论} 
\newtheorem*{zhengming}{证明}
\newtheorem*{wenti}{问题}
\newtheorem*{jieshi}{解释}
\newtheorem{biancheng}{编程}

\renewcommand{\proofname}{证明}

\begin{document}

\title{C语言}
\subtitle{第四次上机\\ 字符串简介及格式化输入输出}
\author{张晓平}
\institute{武汉大学数学与统计学院}


\begin{frame}[plain]\transboxout
\titlepage
\end{frame}

\begin{frame}\transboxin
\begin{center}
\tableofcontents[]%hideallsubsections]
\end{center}
\end{frame}

\AtBeginSection[]{
\begin{frame}[allowframebreaks]
\tableofcontents[currentsection,sectionstyle=show/hide]
\end{frame}
}
%\AtBeginSubsection[]{
%\begin{frame}[allowframebreaks]
%\tableofcontents[currentsection,currentsubsection,subsectionstyle=show/shaded/hide]
%\end{frame}
%}

\begin{frame}[fragile] 
\begin{enumerate}\setcounter{enumi}{0} 
\item 编写程序,输入名和姓,然后以“名, 姓”的格式打印。
\end{enumerate}

\begin{lstlisting}[backgroundcolor=\color{red!10}]
Ming Li
\end{lstlisting}
\end{frame}

\begin{frame}[fragile] 
\lstinputlisting[language=c,numbers=left,frame=single]{Code_OP/ex01.c}
\end{frame}

\begin{frame}[fragile] 
\begin{enumerate}\setcounter{enumi}{1} 
\item 编写程序,输入名字,并执行以下操作:\\[0.05in]
\begin{itemize}
\item 把名字括在双引号中打印出来\\[0.1in]
\item 在宽度为20个字符的字段内打印名字,并且整个字段括在引号内\\[0.1in]
\item 在宽度为20个字符的字段的左端打印名字,并且整个字段括在引号内\\[0.1in]
\item 在比名字宽3个字符的字段内打印它。
\end{itemize}
\end{enumerate}

\begin{lstlisting}[showspaces=true,backgroundcolor=\color{red!10}]
"Xiaoping"
"            Xiaoping"
"Xiaoping            "
"   Xiaoping"
\end{lstlisting}
\end{frame}

\begin{frame}[fragile,allowframebreaks] 
\lstinputlisting[language=c,numbers=left,frame=single]{Code_OP/ex02.c}
\end{frame}

\begin{frame}[fragile] 
\begin{enumerate}\setcounter{enumi}{2} 
\item 编写程序,读取一点浮点数,以如下方式打印:
\begin{lstlisting}[showspaces=true,backgroundcolor=\color{red!10}]
a. The input is 21.3 or 2.1e+01
b. The input is +21.290 or 2.129E+01
\end{lstlisting}
\end{enumerate}
\end{frame}

\begin{frame}[fragile,allowframebreaks] 
\lstinputlisting[language=c,numbers=left,frame=single]{Code_OP/ex03.c}
\end{frame}

\begin{frame}[fragile] 
\begin{enumerate}\setcounter{enumi}{3} 
\item 编写程序,要求输入身高(以cm为单位)和名字,然后以如下形式显示:
\begin{lstlisting}[showspaces=true,backgroundcolor=\color{red!10}]
Xiaoping, you are 1.70m tall.
\end{lstlisting}
使用float类型,使用/作为除号。
\end{enumerate}
\end{frame}

\begin{frame}[fragile,allowframebreaks] 
\lstinputlisting[language=c,numbers=left,frame=single]{Code_OP/ex04.c}
\end{frame}

\begin{frame}[fragile] 
\begin{enumerate}\setcounter{enumi}{4} 
\item 编写程序,首先输入名字,然后输入姓氏。在一行打印输入的姓名,在下一行打印每个名字中字母的个数,把字母个数与相应名字的结尾对齐。以如下形式显示:
\begin{lstlisting}[showspaces=true]
Xiaoping Zhang
       8     5
\end{lstlisting}
然后打印相同的信息,但是字母个数与相应单词的开始对齐。
\end{enumerate}
\end{frame}

\begin{frame}[fragile,allowframebreaks] 
\lstinputlisting[language=c,numbers=left,frame=single]{Code_OP/ex05.c}
\end{frame}

\begin{frame}[fragile] 
\begin{enumerate}\setcounter{enumi}{5} 
\item 编写程序,设置一个值为1.0/3.0的double类型变量和一个值为1.0/3.0的float类型变量。每个变量的值显示三次:\\[0.05in]
\begin{itemize}
\item 一次在小数点后显示4位;\\[0.1in]
\item 一次在小数点后显示12位;\\[0.1in]
\item 一次在小数点后显示16位。\\[0.05in]
\end{itemize}
同时让程序包含float.h文件,并显示FLT\_DIG和DBL\_DIG的值。
\end{enumerate}
\end{frame}

\begin{frame}[fragile,allowframebreaks] 
\lstinputlisting[language=c,numbers=left,frame=single]{Code_OP/ex06.c}
\end{frame}


\end{document}
