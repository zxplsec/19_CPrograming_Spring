\section{地址运算符:\&}
\begin{frame}[fragile]\ft{\secname}
C最重要的、也是最复杂的一个概念是指针(pointer),即\textcolor{acolor1}{用来存储地址的变量}。

\end{frame}

\begin{frame}[fragile]\ft{\secname}
\begin{itemize}
\item
scanf函数使用地址作为参数。\\[0.1in]
\item
更一般地,若想在无返回值的被调函数中修改调用函数的某个数据,必须使用地址参数。
\end{itemize}
\end{frame}

\begin{frame}[fragile]\ft{\secname}
\&为单目运算符,可以取得变量的存储地址。 
\pause 
\vspace{0.4in}

设var为一个变量,则\&var为该变量的地址。

\vspace{0.1in}

\textcolor{acolor1}{一个变量的地址就是该变量在内存中的地址。} 

\end{frame}

\begin{frame}[fragile]\ft{\secname}
设有如下语句
\begin{lstlisting}
var = 24;
\end{lstlisting}
并假定var的存储地址为07BC,则执行语句
\begin{lstlisting}
printf("%d %p\n", var, &var);
\end{lstlisting}
的结果为
\begin{lstlisting}
24 07BC
\end{lstlisting}
\end{frame}

\begin{frame}[fragile,allowframebreaks]\ft{\secname}
  \lstinputlisting
  [language=c,numbers=left,frame=single]
  {Code/loccheck.c}
\end{frame}


\begin{frame}[fragile]\ft{\secname}
\begin{lstlisting}[backgroundcolor=\color{red!10}]
In main(), var1 =  2 and &var1 = 0x7fff5fbff7d8
In main(), var2 =  5 and &var2 = 0x7fff5fbff7d4
In func(), var1 = 10 and &var1 = 0x7fff5fbff7a8
In func(), var2 =  5 and &var2 = 0x7fff5fbff7ac
\end{lstlisting}
\end{frame}

\begin{frame}[fragile]\ft{\secname}
\begin{itemize}
\item 两个{\tf var1}变量具有不同的地址,两个{\tf var2}变量也是如此。\\[0.1in]
\item 调用{\tf func}函数时,把实参({\tf main}函数中的{\tf var2})的值5传递给了形参({\tf func}函数中的{\tf var2})。但这种传递只是进行了数值传递,两个变量仍是独立的。
\end{itemize}
\end{frame}


