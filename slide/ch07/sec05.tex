\section{条件运算符}
\begin{frame}[fragile]\ft{\secname}
C提供一种简写方式来表示{\tf if else}语句,被称为条件表达式,并使用条件运算符{(\tf ? :)}。它是C语言中唯一的三目操作符。
\end{frame}

\begin{frame}[fragile]\ft{\secname}
\begin{lstlisting}[title=求绝对值]
x = (y < 0) ? -y : y;
\end{lstlisting}
\rule{\textwidth}{.8mm} \pause \vspace{.05in}

\begin{itemize}
\item
含义:若y小于0,则x = -y;否则,x = y。\\[0.1in]
\item 用if else描述为
\begin{lstlisting}
if (y < 0)
  x = -y;
else
  x = y;  
\end{lstlisting}
\end{itemize}
\end{frame}

\begin{frame}[fragile]\ft{\secname}
\begin{lstlisting}[title=条件表达式的语法]
expresion1 ? expression2 : expression3
\end{lstlisting}
\rule{\textwidth}{.8mm} \pause \vspace{.05in}

若expresion1为真,则条件表达式的值等于expression2的值;
若expresion1为假,则条件表达式的值等于expression3的值。
\end{frame}

\begin{frame}[fragile]\ft{\secname}
若希望将两个可能的值中的一个赋给变量时,可使用条件表达式。典型的例子是将两个值中的最大值赋给变量:
\begin{lstlisting}[frame=single]
max = (a > b) ? a : b;
\end{lstlisting}
\end{frame}

\begin{frame}[fragile]\ft{\secname}
if else语句能完成与条件运算符同样的功能。但是,条件运算符语句更简洁;并且可以产生更精简的程序代码。
\end{frame}

\begin{frame}[fragile]\ft{\secname}
\begin{li}
设每罐油漆可喷200平方英尺,编写程序计算向给定的面积喷油漆,全部喷完需要多少罐油漆。
\end{li}
\end{frame}

\begin{frame}[fragile]\ft{\secname}
\lstinputlisting[language=c,numbers=left,frame=single]{Code/paint.c}

\end{frame}

\begin{frame}[fragile]\ft{\secname}
\begin{lstlisting}[backgroundcolor=\color{blue!20}]
Enter number of square feet to be painted:
200
You need 1 can of paint.
Enter next value (q to quit):
225
You need 2 cans of paint.
Enter next value (q to quit):
q
\end{lstlisting}
\end{frame}


