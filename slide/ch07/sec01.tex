\section{if语句}

\begin{frame}[fragile,allowframebreaks]\ft{\secname}
\lstinputlisting
[language=c,numbers=left,frame=single]
{Code/colddays.c}
\end{frame}


\begin{frame}[fragile]\ft{\secname}
\begin{lstlisting}[backgroundcolor=\color{red!10}]
Enter the list of daily low temperature.
Use Celsius, and enter q to quit.
-10 -5 0 12 5 6 -4 8 -2 15
q
10 days total: 40.0% were below freezing.
\end{lstlisting}
\end{frame}

\begin{frame}[fragile]\ft{\secname}
if语句被称为分支语句,其一般形式为
\begin{lstlisting}[language=c]
if (condition)
  statement
  
if (condition){
  statements
}  
\end{lstlisting}
\begin{itemize}
\item
若condition的值为真,则执行statements;否则跳过该语句。\\[0.1in]
\item
if结构和while结构相似,主要区别在于,在if结构中,判断和执行仅有一次,而在while结构中,判断和执行可以重复多次。
\end{itemize}
\end{frame}

\begin{frame}[fragile]\ft{\secname}
\begin{itemize}
\item
condition是一个关系表达式,通常是比较两个量的大小。更一般地,condition可以是任何表达式,其值为0就被视为假。\\[0.1in]
\item
语句部分可以是一条简单语句,也可以是一个由花括号括起的复合语句:
\begin{lstlisting}[]
if (score >= 60)
  printf("Pass!\n");
  
if (a > b){
  a++;
  printf("You lose. b.\n");
}  
\end{lstlisting}
\end{itemize}

\end{frame}
