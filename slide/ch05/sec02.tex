\section{其它运算符}

\subsection{取模运算符 {\tf \%}}
\begin{frame}[fragile]\ft{取模运算符 {\tf \%}}
取模运算符(modulus operator)用于计算整数相除所得的余数,\textcolor{acolor1}{只适用于整数运算}。

\end{frame}

\begin{frame}[fragile]\ft{取模运算符 {\tf \%}}
\lstinputlisting[language=c,numbers=left,frame=single]{Code/modulus.c}

\begin{lstlisting}[backgroundcolor=\color{red!10}]
$ gcc modulus.c
$ ./a.out  
13 % 5 = 3
\end{lstlisting}
\end{frame}

\begin{frame}[fragile,allowframebreaks]\ft{取模运算符 {\tf \%}}
\lstinputlisting[language=c,numbers=left,frame=single]{Code/min_sec.c}
\end{frame}


\begin{frame}[fragile]\ft{取模运算符 {\tf \%}}
\begin{lstlisting}[backgroundcolor=\color{red!10}]
Convert seconds to minutes and seconds!
Enter the number of seconds (<=0 to quit):
154
154 seconds is 2 minutes, 34 seconds.
Enter next value (<=0 to quit):
567
567 seconds is 9 minutes, 27 seconds.
Enter next value (<=0 to quit):
0
Done!
\end{lstlisting}
\end{frame}

\begin{frame}[fragile,allowframebreaks]\ft{负数的取模运算}
\lstinputlisting[language=c,numbers=left,frame=single]{Code/mod_negative.c}

\end{frame}

\begin{frame}[fragile]\ft{负数的取模运算}
\begin{lstlisting}[backgroundcolor=\color{red!10}]
 11 /  5 =  2,  11 %  5 =  1
 11 / -5 = -2,  11 % -5 =  1
-11 / -5 =  2, -11 % -5 = -1
-11 /  5 = -2, -11 %  5 = -1
\end{lstlisting}
\end{frame}

\begin{frame}[fragile]\ft{负数的取模运算}
\begin{itemize}
\item C99规定,整数除法依“趋零截尾”的原则。\\[0.1in]
\item 对于取模运算,
  \begin{itemize}
  \item 若第一个操作数为负数,则得到的模也为负数;
  \item 若第一个操作数为整数,则得到的模也为正数。
  \end{itemize}
\end{itemize}
\end{frame}

\subsection{自增自减运算符}
\begin{frame}[fragile]\ft{自增运算符 {\tf ++}}
自增运算符(increment operator)使其操作数的值增加1。\vspace{0.1in}

\begin{itemize}
\item 前缀模式:
\begin{lstlisting}[backgroundcolor=\color{red!10}]
++i;
\end{lstlisting} 
\item 后缀模式:
\begin{lstlisting}[backgroundcolor=\color{red!10}]
i++;
\end{lstlisting} 
\end{itemize}
两种模式的相似之处在于都使操作数自增1,区别在于自增这一动作发生的时间不同。
\end{frame}

\begin{frame}[fragile]\ft{两种模式的相似之处}
  \lstinputlisting[language=c,numbers=left,frame=single]{Code/add_one.c}
\end{frame}

\begin{frame}[fragile]\ft{两种模式的相似之处}  
\begin{lstlisting}[backgroundcolor=\color{red!10}]
i = 1, j = 1
i = 2, j = 2
i = 3, j = 3
i = 4, j = 4
i = 5, j = 5
\end{lstlisting}    
\end{frame}


\begin{frame}[fragile]\ft{两种模式的相似之处}
\begin{lstlisting}[backgroundcolor=\color{red!10}]
++i;
j++;
\end{lstlisting}
可以替换为
\begin{lstlisting}[backgroundcolor=\color{red!10}]
i = i + 1;
j = j + 1;
\end{lstlisting} \pause 

\textcolor{acolor1}{单独使用自增运算符时,前缀模式与后缀模式效果相同。}
\end{frame}

\begin{frame}[fragile]\ft{为什么会创建自增运算符?}

使程序更为简洁,可读性更强

\begin{lstlisting}[backgroundcolor=\color{red!10}]
shoe = 3.0;
while (shoe < 18.5)
{
  foot = SCALE*size + ADJUST;
  printf("%10.1f %20.2f inches\n", shoe, foot);
  ++shoe;
}
\end{lstlisting}
\end{frame}

\begin{frame}[fragile]\ft{为什么会创建自增运算符?}
进一步简化:
\begin{lstlisting}[backgroundcolor=\color{red!10}]
shoe = 3.0;
while (++shoe < 18.5)
{
  foot = SCALE*size + ADJUST;
  printf("%10.1f %20.2f inches\n", shoe, foot);
}
\end{lstlisting}
\end{frame}

\begin{frame}[fragile]\ft{前缀与后缀模式的不同}
\lstinputlisting[language=c,numbers=left,frame=single]{Code/post_pre.c}    
\end{frame}

\begin{frame}[fragile]\ft{前缀与后缀模式的不同}
  \begin{lstlisting}[backgroundcolor=\color{red!10}][backgroundcolor=\color{red!10}]
$ gcc post_pre.c
$ ./a.out    
a = 2, aplus = 1
b = 2, bplus = 2
\end{lstlisting}    
\end{frame}



\begin{frame}[fragile]\ft{前缀与后缀模式的不同}
\begin{lstlisting}[backgroundcolor=\color{red!10}]
shoe = 3.0;
while (shoe++ < 18.5)
{
  foot = SCALE * size + ADJUST;
  printf("%10.1f %20.2f inches\n", shoe, foot);
}
\end{lstlisting}
\end{frame}

\begin{frame}[fragile]\ft{前缀与后缀模式的不同}
\begin{itemize}
\item
在使用自增运算符时,请自问一下是否能互换前缀和后缀模式?\\[0.1in]
\item
一个明智的选择是避免那些两种模式将导致不同效果的代码。例如,不要使用
\begin{lstlisting}[backgroundcolor=\color{red!10}]
b = ++i;
\end{lstlisting}
可用以下语句代替:
\begin{lstlisting}[backgroundcolor=\color{red!10}]
++i;
b = i;
\end{lstlisting}
\item 然后有时不那么谨慎会更有趣。
\end{itemize}
\end{frame}

\begin{frame}[fragile]\ft{两种模式的不同}
观察代码
\begin{lstlisting}[backgroundcolor=\color{red!10}]
i = 5;
b = ++i;
\end{lstlisting}
\begin{lstlisting}[backgroundcolor=\color{red!10}]
i = 5;
b = i++;
\end{lstlisting}
请分别指出执行后b和i的值?
\end{frame}

\begin{frame}[fragile]\ft{自减运算符 {\tf --}}
\begin{lstlisting}[backgroundcolor=\color{red!10}]
--count; //`自减运算符的前缀模式`

count--; //`自减运算符的后缀模式`
\end{lstlisting}
\end{frame}

\begin{frame}[fragile]\ft{自增和自减运算符的优先级}
自增和自减运算符有很高的优先级,只有圆括号比它们的优先级高。如
\begin{lstlisting}[backgroundcolor=\color{red!10}]
x * y++
\end{lstlisting}
等价于
\begin{lstlisting}[backgroundcolor=\color{red!10}]
x * (y++)
\end{lstlisting}
而不是
\begin{lstlisting}[backgroundcolor=\color{red!10}]
(x * y)++   // invalid
\end{lstlisting}
\textcolor{acolor1}{自增和自减运算符只能作用于变量。}
\end{frame}

\begin{frame}[fragile]\ft{自增和自减运算符的优先级}
不要将自增和自减运算符的\textcolor{acolor3}{优先级和求值顺序}弄混淆。
\end{frame}

\begin{frame}[fragile]\ft{自增和自减运算符的优先级}
\lstinputlisting[language=c,numbers=left,frame=single]{Code/inc.c}  
\end{frame}

\begin{frame}[fragile]\ft{自增和自减运算符的优先级}
\begin{lstlisting}[backgroundcolor=\color{red!10}][backgroundcolor=\color{red!10}]
`中间过程:`nextnum = (2 + 3) * 6 = 5 * 6 = 30 
`运行结果:`n = 4, nextnum = 30
\end{lstlisting}
\end{frame}

\begin{frame}[fragile]\ft{自增和自减运算符的优先级}
\begin{itemize}
\item \tf n++表示在使用n之后,n的值才自增。\\[0.1in]
\item 优先级告诉我们++只属于n,也告诉我们什么时候使用n的值。\\[0.1in]
\item 而自增运算符的性质决定了什么时候改变n的值。
\end{itemize}
\end{frame}

\begin{frame}[fragile]\ft{自增和自减运算符的优先级}
\lstinputlisting[language=c,numbers=left,frame=single]{Code/inc1.c}
\end{frame}

\begin{frame}[fragile]\ft{自增和自减运算符的优先级}
\begin{lstlisting}[backgroundcolor=\color{red!10}]
`中间过程:`nextnum = (2 + 4)*6 = 6*6 = 36 
`运行结果:`n = 4, nextnum = 36
\end{lstlisting}
\end{frame}

\begin{frame}[fragile]\ft{自增和自减运算符的优先级}
\begin{itemize}
\item \tf 当n++是表达式的一部分时,它表示:\textcolor{acolor1}{先使用n,然后将它的值增加}\\[0.1in]
\item 当++n是表达式的一部分时,它表示:\textcolor{acolor1}{先将n的值增加,然后再使用它}。
\end{itemize}
\end{frame}

\begin{frame}[fragile]\ft{Don't Be Too Clever}
  \lstinputlisting[language=c,numbers=left,frame=single]{Code/inc2.c}
\end{frame}

\begin{frame}[fragile]\ft{Don't Be Too Clever}
\begin{lstlisting}[backgroundcolor=\color{red!10}]
maybe:    n = 5, n^2 = 25
maybe:    n = 6, n^2 = 25
maybe:    n = 6, n^2 = 30
\end{lstlisting}
\end{frame}

\begin{frame}[fragile]\ft{Don't Be Too Clever}
\tf C编译器可以选择先计算函数中哪个参数的值。这个自由提高了编译器的效率,但若在函数参数里使用自增自减运算符就会带来麻烦。
\end{frame}

\begin{frame}[fragile]\ft{Don't Be Too Clever}
  \lstinputlisting[language=c,numbers=left,frame=single]{Code/inc3.c}
\end{frame}

\begin{frame}[fragile]\ft{Don't Be Too Clever}
\tf 编译器可能从左到右依次计算,也可能从右到左依次计算,这些都可能导致不可预知的结果。
\end{frame}

\begin{frame}[fragile]\ft{Don't Be Too Clever}
\lstinputlisting[language=c,numbers=left,frame=single]{Code/inc4.c}
\end{frame}

\begin{frame}[fragile]\ft{Don't Be Too Clever}
\begin{itemize}
\item\tf 执行后,n的值为5,但m的值不确定。\\[0.1in]
\item 有的编译器计算m时使用n的旧值两次,然后将n增加两次,从而使m的值为6,n的值为5。\\[0.1in]
\item 有的编译器计算m时使用n的旧值一次,然后增加n的值一次,再使用第二个n的值,最后第二次增加n的值。此方法使m的值为7,n的值为5。
\end{itemize}
\end{frame}

\begin{frame}[fragile]\ft{请使用以下原则}
\begin{itemize}
\item 若一个变量出现在同一个函数的多个参数中,不要对它使用自增或自减运算符。\\[0.1in]
\item 当一个变量多次出现在一个表达式时,不要对它使用自增或自减运算符。\end{itemize}
\end{frame}
