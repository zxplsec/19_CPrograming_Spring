\section{C与C++中的逗号}
\begin{frame}[fragile]\ft{\secname}
  在C 与 C++中,逗号有两层含义:
  \begin{enumerate}
\item[1.] 逗号充当运算符。
\item[] 逗号运算符为一元运算符,先计算第一个操作数并舍弃之,然后计算第二个操作数并返回该值。逗号运算符具有最低优先级,并且是一个顺序点。
\end{enumerate}
\end{frame}

\begin{frame}[fragile]\ft{\secname}
    \begin{lstlisting}[language=c,backgroundcolor=\color{red!10}]
/* comma as an operator */
int i = (5, 10);  /* 10 is assigned to i*/
int j = (f1(), f2());
 /* f1() is called (evaluated) first followed by f2(). The returned value of f2() is assigned to j */    
  \end{lstlisting}
\end{frame}

\begin{frame}[fragile]\ft{\secname}
 \begin{enumerate}
\item[2.] 逗号充当分隔符
\item[] 逗号作为分隔符,通常用于函数调用与定义,函数宏,变量声明,enum声明以及结构体中。
  \begin{lstlisting}[language=c,backgroundcolor=\color{red!10}]
/* comma as a separator */
int a = 1, b = 2;
void fun(x, y);    
  \end{lstlisting}
\end{enumerate}
\end{frame}

%% %% The use of comma as a separator should not be confused with the use as an operator. For example, in below statement, f1() and f2() can be called in any order.

\begin{frame}[fragile]\ft{\secname}

\begin{lstlisting}[language=c,backgroundcolor=\color{red!10}]
/* Comma acts as a separator here and doesn't enforce any sequence. 
    Therefore, either f1() or f2() can be called first */
void fun(f1(), f2());  
\end{lstlisting}
\end{frame}
%% %% You can try below programs to check your understanding of comma in C.

\begin{frame}[fragile]\ft{\secname}
  \lstinputlisting[language=c,backgroundcolor=\color{red!10},numbers=left]{Code/comma1.c}
\end{frame}
\begin{frame}[fragile]\ft{\secname}
  \lstinputlisting[language=c,backgroundcolor=\color{red!10},numbers=left]{Code/comma2.c}
\end{frame}
\begin{frame}[fragile,allowframebreaks]\ft{\secname}
  \lstinputlisting[language=c,backgroundcolor=\color{red!10},numbers=left]{Code/comma3.c}
\end{frame}
 
