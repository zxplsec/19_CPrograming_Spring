\section{C与C++中的逗号}
\begin{frame}[fragile]\ft{\secname}
  在 C 与 C++ 中,逗号 \lstinline|','| 有两层含义:
  \begin{enumerate}
  \item[1.] 充当运算符。
  \item[] 为一元运算符,先计算第一个操作数并舍弃之,然后计算第二个操作数并返回该值。逗号运算符具有最低优先级,并且是一个顺序点。
    \pause
  
  \begin{lstlisting}[language=c,backgroundcolor=\color{red!10}]
int i = (5, 10);   // i = 10
int j = (f1(), f2()); // j = f2()
  \end{lstlisting} \pause 
% \end{frame}

% \begin{frame}[fragile]\ft{\secname}
\item[2.] 充当分隔符
\item[] 通常用于函数调用与定义,函数宏,变量声明,\lstinline|enum| 声明以及结构体中。 \pause 
  \begin{lstlisting}[language=c,backgroundcolor=\color{red!10}]
int a = 1, b = 2;
void fun(x, y);    
  \end{lstlisting}
\end{enumerate}
\end{frame}

%% %% The use of comma as a separator should not be confused with the use as an operator. For example, in below statement, f1() and f2() can be called in any order.

\begin{frame}[fragile]\ft{\secname}

\begin{lstlisting}[language=c,backgroundcolor=\color{red!10}]
  void fun(f1(), f2());  
\end{lstlisting}
当逗号充当分隔符时,不要与运算符混淆。因此,\lstinline|f1()| 与 \lstinline|f2()| 的调用次序是任意的。
\end{frame}
%% %% You can try below programs to check your understanding of comma in C.

\begin{frame}[fragile]\ft{\secname}
  \lstinputlisting
  {slide05/code/comma1.c}
  \pause
  \begin{lstlisting}[backgroundcolor=\color{red!20}]
15    
  \end{lstlisting}
  
\end{frame}
\begin{frame}[fragile]\ft{\secname}
  \lstinputlisting
  {slide05/code/comma2.c}
    \pause
  \begin{lstlisting}[backgroundcolor=\color{red!20}]
12    
  \end{lstlisting}
\end{frame}

\begin{frame}[fragile]\ft{\secname}
  \lstinputlisting
  {slide05/code/comma3.c}

  \pause
  \begin{lstlisting}[backgroundcolor=\color{red!20}]
x = 11
x = 12
y = 12
x = 13
  \end{lstlisting}

\end{frame}
 
