\section{关系运算符与逻辑运算符}

\subsection{关系运算符}

\begin{frame}\ft{\subsecname}
关系运算符用于比较两个值。
\begin{enumerate}
\item
  运算符 {\tf ==}  检查两个给定的操作数是否相等。若相等,返回{\tf true};否则返回{\tf false}。如 {\tf 5 == 5} 返回{\tf true}。
\item
  运算符 {\tf !=}  检查两个给定的操作数是否相等。若不相等,返回{\tf true};否则返回{\tf false}。如 {\tf 5 != 5} 返回{\tf false}。
\end{enumerate}
\end{frame}

\begin{frame}\ft{\subsecname}
  \begin{enumerate}
\item[3.]
  运算符 {\tf >} 检查第一个操作数是否大于第二个操作数。若成立,返回{\tf true};否则返回{\tf false}。如 {\tf 6 > 5} 返回{\tf true}。
\item[4.]
  运算符 {\tf <} 检查第一个操作数是否小于第二个操作数。若成立,返回{\tf true};否则返回{\tf false}。如 {\tf 6 < 5} 返回{\tf false}。
\end{enumerate}
\end{frame}

\begin{frame}\ft{\subsecname}
  \begin{enumerate}
\item[5.]
  运算符 {\tf >=} 检查第一个操作数是否大于或等于第二个操作数。若成立,返回{\tf true};否则返回{\tf false}。如 {\tf 5 >= 5} 返回{\tf true}。
\item[6.]
  运算符 {\tf <=} 检查第一个操作数是否小于或等于第二个操作数。若成立,返回{\tf true};否则返回{\tf false}。如 {\tf 5 <= 5} 返回{\tf true}。
\end{enumerate}
  
\end{frame}

\begin{frame}[fragile,allowframebreaks]\ft{\subsecname}
\lstinputlisting[language=c,backgroundcolor=\color{red!10},numbers=left]{Code/rel_operand.c}
  
\end{frame}


\begin{frame}[fragile]\ft{\subsecname}
\begin{lstlisting}[backgroundcolor=\color{red!10}]
Output:
a is greater than b
a is greater than or equal to b
a is greater than or equal to b
a is greater than b
a and b are not equal
a is not equal to b  
\end{lstlisting}  
\end{frame}


\subsection{逻辑运算符}
\begin{frame}
  逻辑运算符用于连接两个及以上条件,或对原条件取否。%%% They are used to combine two or more conditions/constraints or to complement the evaluation of the original condition in consideration.
\begin{enumerate}
\item
  \textcolor{acolor3}{逻辑与}: 当两个条件同时满足时,运算符 {\tf \&\&} 返回{\tf true};否则返回 {\tf false}。如,当 {\tf a} 和 {\tf b} 均为 {\tf true} (即非零)时,{\tf a \&\& b} 返回{\tf true}。
\item
  \textcolor{acolor3}{逻辑或}:当至少有一个条件满足时,运算符 {\tf ||} 返回{\tf true};否则返回 {\tf false}。如,当 {\tf a} 和 {\tf b} 至少有一个为 {\tf true} (即非零)时,{\tf a || b} 返回{\tf true}。当然,当 {\tf a} 和 {\tf b} 均为 {\tf true} 时, {\tf a || b}返回{\tf true}。
\item
  \textcolor{acolor3}{逻辑非}:当条件不满足时,运算符 {\tf !} 返回 {\tf true} ;否则返回 {\tf false} 。如,若 {\tf a} 为 {\tf false} 时,{\tf a} 返回 {\tf true}。
\end{enumerate}
  
\end{frame}

\begin{frame}[fragile,allowframebreaks]\ft{\subsecname}
\lstinputlisting[language=c,backgroundcolor=\color{red!10},numbers=left]{Code/logic_operand.c}
  
\end{frame}


\begin{frame}[fragile]\ft{\subsecname}

\begin{lstlisting}[backgroundcolor=\color{red!10}]
AND condition not satisfied
a is greater than b OR c is equal to d
a is not zero  
\end{lstlisting}
\end{frame}

\subsection{逻辑运算符中的短路现象}

\begin{frame}[fragile,allowframebreaks]\ft{\subsecname} 
对于逻辑与,若第一个操作数为 {\tf false} ,则第二个操作数将不会被计算。如以下程序将不会打印 {\tf Hello}。
\lstinputlisting[language=c,backgroundcolor=\color{red!10},numbers=left]{Code/logic_short_circuit1.c}

\end{frame}

\begin{frame}[fragile]\ft{\subsecname} 
但下面的程序将打印 {\tf Hello}。
\lstinputlisting[language=c,backgroundcolor=\color{red!10},numbers=left]{Code/logic_short_circuit2.c}
\end{frame}

\begin{frame}[fragile]\ft{\subsecname} 
对于逻辑或,若第一个操作数为 {\tf true} ,则第二个操作数不会被计算。如以下程序不会打印 {\tf Hello}。
\lstinputlisting[language=c,backgroundcolor=\color{red!10},numbers=left]{Code/logic_short_circuit3.c}

\end{frame}

\begin{frame}[fragile]\ft{\subsecname} 

但下面的程序将打印 {\tf Hello}。
\lstinputlisting[language=c,backgroundcolor=\color{red!10},numbers=left]{Code/logic_short_circuit4.c}

\end{frame}
