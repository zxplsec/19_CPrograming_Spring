\section{基本概念}

%\begin{frame}
%数据类型有两大系列:
%\begin{enumerate}
%\item 整数类型
%\item 浮点数类型
%\end{enumerate}
%
%本章将介绍这些数据类型以及如何声明它们、如何使用它们。
%\end{frame}

\begin{frame}\ft{常量}
\begin{dingyi}\textcolor{acolor3}{常量}

在程序执行过程中,其值不发生改变的量称为常量。
\end{dingyi} \vspace{0.1in}

常量分为两类:\vspace{0.05in}

\begin{enumerate}
\item 直接常量(或字面常量)\\[0.1in]
\item 符号常量
\end{enumerate}
\end{frame}


\begin{frame}\ft{直接常量}

\begin{itemize}
\item 整型常量:12、0、-3;\\[0.1in]
\item 浮点型常量:3.1415、-1.23;\\[0.1in]
\item 字符型常量:'a'、'b'
\end{itemize}
\end{frame}

\begin{frame}[fragile]\ft{符号常量}

\begin{dingyi}\textcolor{acolor3}{标识符}

用来标识变量名、符号常量名、函数名、数组名、类型名、文件名的有效字符序列。
\end{dingyi} \pause \vspace{0.1in}

\begin{dingyi}\textcolor{acolor3}{符号常量}

在C语言中,可以用一个标识符来表示一个常量,称之为符号常量。
\end{dingyi} 
\end{frame}

\begin{frame}[fragile]\ft{符号常量}
符号常量在使用之前必须先定义,其一般形式为:
\begin{lstlisting}[language=C]
#define `标识符` `常量`
\end{lstlisting}
\vspace{0.05in}

\begin{itemize}
\item \#define是一条预处理命令,称为宏定义命令。\\[0.1in]
\item 功能是把该标识符定义为其后的常量值。\\[0.1in]
\item 一经定义,以后在程序中所有出现该标识符的地方均代之以该常量值。
\end{itemize}
\end{frame}

\begin{frame}[fragile]\ft{符号常量}
\begin{lstlisting}[language=c,title=price.c,frame=single,numbers=left]
#include<stdio.h>
#define PRICE 100
int main(void)
{
  int num, total;  
  num = 10;
  total = num * PRICE;
  printf("total=%d\n", total);  
  return 0;
}
\end{lstlisting}

\end{frame}

\begin{frame}[fragile]\ft{符号常量}
使用符号常量的好处是:\vspace{0.05in}

\begin{itemize}
\item 含义清楚;\\[0.1in]
\item 能做到“一改全改”。
\end{itemize}
\end{frame}


\begin{frame}\ft{变量}
\begin{dingyi}\textcolor{acolor3}{变量}

在程序执行过程中,其值可以改变的量称为变量。
\end{dingyi}

\begin{itemize}
\item
一个变量应该有一个名字,在内存中占据一定的存储单元。\\[0.1in]
\item
变量定义必须放在变量使用之前。
\end{itemize}
\end{frame}


\begin{frame}\ft{变量}
\begin{itemize}
\item
一般放在函数体的开头部分。\\[0.1in]
\item
要区分变量名和变量值是两个不同的概念。
\end{itemize}

\begin{figure}
\centering
\begin{tikzpicture}
\draw[thick] (0,0) rectangle (1,1);
\node at (0.5,0.5) [] {3};
\node at (0.5,1.3) [] {$a$};
\draw[->,>=stealth] (1.5,1.3) node[right]{\footnotesize{变量名}}--(0.7,1.3);
\draw[->,>=stealth] (1.5,0.5) node[right]{\footnotesize{变量值}}--(0.7,0.5);
\draw[->,>=stealth] (1.5,-.3) node[right]{\footnotesize{存储单元}}--(1.0,-.3)--(0.7,-0.05);
\end{tikzpicture}
\end{figure}

\end{frame}


\begin{frame}\ft{数据类型}
\begin{figure}
\includegraphics[width=3.5in]{Fig/datatype}
\end{figure}
\end{frame}


\begin{frame}\ft{数据类型}
\begin{itemize}
\item 
对于常量,编译器通过书写形式来辨认其类型。\\[0.1in]
\item[] 例如,42是整型,42.0是浮点型。\\[0.2in]
\item
变量必须在声明语句中指定其类型。
\end{itemize}
\end{frame}

\begin{frame}\ft{数据类型关键字}
\begin{table}
\centering
\begin{tabular}{p{2cm}|p{2cm}|p{2cm}}\hline
int & signed & \_Bool \\[0.05in]
long & void & \_Complex \\[0.05in]
short & & \_Imaginary\\[0.05in]
unsigned &&\\[0.05in]
char &&\\[0.05in]
float &&\\[0.05in]
double &&\\\hline
\end{tabular}
\end{table}
\end{frame}

\begin{frame}\ft{数据类型关键字}
\begin{itemize}
\item 
int提供基本整型,long、short、unsigned和signed为其变种。\\[0.1in]
\item
char用于表示字母及其他字符(如\#、\$、\%、*等),也可表示小的整数。\\[0.1in]
\item
float、double和long double表示浮点型数。\\[0.1in]
\item \_Bool表示布尔值(true和false)。\\[0.1in]
\item \_Complex和\_Imaginary分别表示复数和虚数。\\[0.2in]
\end{itemize}
\textcolor{acolor1}{这些类型按其存储方式被分为两类:整型和浮点型。}
\end{frame}





