\section{程序解释}
\begin{frame}\ft{\#include指示和头文件}

\lstinputlisting[
language=c,
linerange={2-2},
firstnumber=2,
numbers=left,
frame=tb
]{Code/first.c}

\begin{itemize}
\item
相当于在此处复制文件stdio.h的完整内容,以方便在多个程序间共享公用信息。\\[0.1in]
\item
\#include语句是C预处理器指令的一个例子。通常,C编译器在编译前要对源代码做一些准备工作,这称为预处理。
\\[0.1in]
\item
stdio.h文件包含了有关输入和输出函数的信息,以供编译器使用。
\end{itemize}
\end{frame}

\begin{frame}\ft{main函数}
\lstinputlisting[
language=c,
linerange={3-3},
firstnumber=3,
numbers=left,
frame=tb
]{Code/first.c}
 
C程序至少包含一个函数,函数是C程序的基本模块。\\[0.1in]
\begin{table}
\centering
\begin{tabular}{ll}
表达式 & 含义 \\ \hline
(~) & main是一函数名\\[0.1in]
int & main函数返回一整数\\[0.1in]
void & main函数不接受任何参数 \\ \hline
\end{tabular}
\end{table}
 
注:main函数是任何C程序的唯一入口。
\end{frame}

\begin{frame}[fragile]\ft{main函数}
main()函数有三种定义:
\begin{lstlisting}[language=c,frame=single]
// Definition 1: NOT RECOMMENDED
void main() { /* ... */ }

// Definition 2
int main() { /* ... */ }

// Definition 3
int main(int argc, char* argv[]) { /* ... */ }
\end{lstlisting}
\end{frame}

\begin{frame}[fragile]\ft{main函数}
考虑main()的两种定义,它们的差别是什么?
\begin{lstlisting}[language=c,frame=single]
int main()
{
   /*  */
   return 0;
}

int main(void)
{
   /*  */
   return 0;
}
\end{lstlisting}
\end{frame}

\begin{frame}[fragile]\ft{main函数}
在C++中,两种没有差别,完全一致。

两种定义在C中都可以,但是第二种定义更好,因它清晰地表明main()在调用时不允许有参量。
在C中,如果一个函数没有指定任何参量,就意味着该函数允许在调用时有任意多个参量或者无参量。
\end{frame}

\begin{frame}[fragile]\ft{main函数}
\begin{lstlisting}[language=c,frame=single]
// Program 1 (Compiles and runs fine in C, but not in C++)
void fun() {  } 
int main(void)
{
    fun(10, "GfG", "GQ");
    return 0;
}
\end{lstlisting}
上述程序能编译和运行,但以下程序编译会失败。
\end{frame}

\begin{frame}[fragile]\ft{main函数}
\begin{lstlisting}[language=c,frame=single]
// Program 2 (Fails in compilation in both C and C++)
void fun(void) {  }
int main(void)
{
    fun(10, "GfG", "GQ");
    return 0;
}
\end{lstlisting}
不同于C,在C++中,以上两个程序在编译时都会失败,因为在C++中,fun()和fun(void)无差别。
\end{frame}

\begin{frame}[fragile]\ft{main函数}

因此, 在C中int main()和int main(void)差别在,前者允许在调用时有任意多个参量,而后者在调用时不能有任何参量。尽管两者在大多数时候无任何差别,但在实践中更推荐使用int main(void)。
\end{frame}

\begin{frame}[fragile]\ft{main函数}
练习:以下C程序的输出是什么?
\end{frame}

\begin{frame}[fragile]\ft{main函数}
\begin{lstlisting}[language=c,frame=single]
// Question 1
#include <stdio.h>
int main()
{
    static int i = 5;
    if (--i){
        printf("%d ", i);
        main(10);
    }
}
\end{lstlisting}
\end{frame}

\begin{frame}[fragile]\ft{main函数}
\begin{lstlisting}[language=c,frame=single]
// Question 2

#include <stdio.h>
int main(void)
{
    static int i = 5;
    if (--i){
        printf("%d ", i);
        main(10);
    }
}
\end{lstlisting}

\end{frame}

\begin{frame}\ft{注释}
 
\lstinputlisting[
language=c,
linerange={4-4},
firstnumber=4,
numbers=left,
frame=tb
]{Code/first.c}
 


\begin{overprint}
\begin{itemize}
\item {\tf /* ... */}之间的内容是程序注释。\\[0.1in]
\item 注释可以让阅读者更容易理解程序。\\[0.1in]
\item 注释可以放在任意位置,甚至和它要解释的语句在同一行。\\[0.1in]
\item 一个较长的注释可以单独放一行,也可以是多行。\\[0.1in]
\item {\tf /* ... */}之间的所有内容都会被编译器忽略。
\end{itemize}
\end{overprint}

\end{frame}

\begin{frame}[fragile]\ft{注释}
\begin{lstlisting}[language=c]
/* `这是有效的C注释`*/

/* `将注释分成两行写`
   `也是可以的` */

/*
   `也可以这样写`
*/

/* `但这是无效的注释,因为没有结束标记`
\end{lstlisting}
\end{frame}

\begin{frame}[fragile]\ft{注释}
C99增加另一种风格的注释,使用{\tf //}符号。 \vspace{0.2in}


\begin{lstlisting}[language=c]
// `这种注释必须限制在一行内`

int n; // `这种注释也可以写在此处`
\end{lstlisting}
\end{frame}

\begin{frame}[fragile]\ft{花括号,程序体和代码块}
 
\begin{lstlisting}[language=c,frame=tb]
{
  ...
}
\end{lstlisting}
 
\begin{itemize}
\item
C函数使用花括号表示函数体的开始和结束。\\[0.2in]
\item
花括号还可以用来把函数中的语句聚集到一个单元或代码块中。
\end{itemize}
\end{frame}

\begin{frame}[fragile]\ft{声明}
\lstinputlisting[
language=c,
linerange={6-6},
firstnumber=6,
numbers=left,
frame=tb
]{Code/first.c}

 
该语句为声明语句(declaration statement),做两件事情:\vspace{0.1in}
\begin{itemize}
\item[(1)]
在内存中为变量num分配了空间。\\[0.1in]
\item[(2)]
int说明变量num的类型(整型)。
\end{itemize} \vspace{0.1in}

\pause 

注意:分号指明该行是C的一个语句。分号是语句的一部分。
 
\end{frame}

\begin{frame}[fragile]\ft{声明}
Ansi C要求必须在一个代码块的开始处声明变量,在这之前不允许其他任何语句。

\begin{lstlisting}[
language=c,
numbers=left,
frame=tb
]
int main(void)
{
  int n;
  int m;
  n = 5;
  m = 3;
  // other statements
}
\end{lstlisting}
\end{frame}

\begin{frame}[fragile]\ft{声明}
C99遵循C++的惯例,允许把声明放在代码块的任何位置。但是在首次使用变量之前仍必须先声明它。

\begin{lstlisting}[
language=c,
numbers=left,
frame=tb
]
int main(void)
{
  int n;
  n = 5;
  // more statements
  int m;
  m = 3;
  // other statements
}
\end{lstlisting}
\end{frame}


\begin{frame}[fragile]\ft{声明}
\begin{wenti}
\begin{itemize}
\item 数据类型是什么?
\item 可以选择什么样的名字?
\item 为什么必须对变量进行声明?
\end{itemize}
\end{wenti}
\end{frame}

\begin{frame}[fragile]\ft{声明}
\begin{enumerate}[1]
\item 数据类型\\[0.1in]
\item[] C可以处理多个数据种类,如整数、字符和浮点数。\\[0.1in]
\item[] 把一个变量声明为整数类型、字符类型或浮点数类型,是计算机正确地存储、获取和解释该数据的基本前提。
\end{enumerate}
\end{frame}

\begin{frame}[fragile]\ft{声明}
\begin{enumerate}[2]
\item 名字的选取\\[0.1in]
\item[] 应尽量使用有意义的变量名。\\[0.1in]
\item[] 若名字不能表达清楚,可以用注释解释变量所代表的意思。\\[0.1in]
\item[] 通过这些方式使程序更易读是良好编程的基本技巧之一。
\end{enumerate}
\end{frame}


\begin{frame}[fragile]\ft{声明}
\textcolor{acolor3}{命名规则:} \vspace{0.1in}
\begin{enumerate}
\item 只能使用字母、数字和下划线,且第一个字符不能为数字。 \\[0.1in]
\item 操作系统和C库通常使用以一个或两个下划线开始的名字,因此最好避免这种用法。 \\[0.1in]
\item C区分大小写,如stars不同于Stars或STARS。

\end{enumerate}

\end{frame}


\begin{frame}[fragile]\ft{声明}

\begin{table}
\centering
\caption{正确和错误的名字}
\begin{tabular}{c|c}\hline
${\checkmark}$&${\texttimes}$\\[0.1in]\hline
wiggles &   \$zj**\\[0.1in]
cat2 &  2cat\\[0.1in]
Hot\_Dog &  Hot-Dog\\[0.1in]
taxRate &  tax Rate\\[0.1in]
\_kcab &  don't\\\hline
\end{tabular}
\end{table}


\end{frame}


\begin{frame}[fragile]\ft{声明}
\begin{enumerate}[3]
\item 声明变量的好处\\[0.1in]
\item[] 把所有变量放在一起,可以让读者更容易掌握程序的内容。
\\[0.1in]
\item[] 在开始编写程序之前,考虑一下需要声明的变量会促使你做一些计划。\\[0.1in]
\item[] 声明变量可以帮助避免程序中出现一类很难发现的细微错误,即变量名的错误拼写。\\[0.1in]
\item[] 若没有声明所有变量,将不能编译C程序。
\end{enumerate}
\end{frame}


\begin{frame}[fragile]\ft{赋值}
\lstinputlisting[
language=c,
linerange={7-7},
firstnumber=7,
numbers=left,
frame=tb
]{Code/first.c}

 
该语句是赋值语句(Assignment statement)。赋值语句是C语言的一种基本操作。\vspace{0.1in}

含义:把值赋给变量num。
 
\end{frame}


\begin{frame}[fragile]\ft{printf函数}
\lstinputlisting[
language=c,
linerange={8-10},
firstnumber=8,
numbers=left,
frame=tb
]{Code/first.c}
 

每行都使用了C的一个标准函数printf,其信息由头文件stdio.h指定。
\vspace{0.1in}

圆括号表明printf为函数名,圆括号内为参数(argument)。这里的参数都是字符串,即双引号之间的内容。
 
\end{frame}


\begin{frame}[fragile]\ft{转义字符}
 
转义字符通常用于代表难以表达或无法键入的字符,以“$\backslash$”开头。
 
\begin{table}
\centering
\begin{tabular}{cc} \hline
转移字符 & 含义 \\ \hline  
$\backslash$n & 换行\\
$\backslash$t & Tab键\\
$\backslash$b & 退格\\
$\backslash$' & 单引号\\
$\backslash$" & 双引号\\
$\backslash\backslash$ & 反斜杠\\
\hline 
\end{tabular}
\end{table}
\end{frame}


\begin{frame}[fragile]\ft{格式化字符串}
格式化字符串,也称占位符,用以指定输出项的数据类型和输出格式,以“\%”开头。

\begin{table}
\centering
\begin{tabular}{p{2cm}|p{8cm}}\hline
占位符 & 含义 \\ \hline  
\%d & 用于输出十进制整数(实际长度)\\[0.1in]
\%c & 输出一个字符\\[0.1in]
\%s & 输出一个字符串\\[0.1in]
\%f & 以小数形式输出实数(整数部分全部输出,小数部分6位)\\
\hline 
\end{tabular}
\end{table}
 
\end{frame}

\begin{frame}[fragile]\ft{return语句}
\lstinputlisting[
language=c,
linerange={11-11},
firstnumber=11,
numbers=left,
frame=tb
]{Code/first.c}

带有返回值的C函数要求使用一个return语句,该语句包含关键字return,后面紧跟要返回的值。

\end{frame}
